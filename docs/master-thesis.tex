% template adapted from https://github.com/jgm/pandoc-templates/blob/master/default.latex

%%% STYLE %%%
\documentclass[12pt,]{article}

%%% PACKAGES %%%

% fonts
\usepackage{lmodern}

% pdf
\usepackage{pdfpages}

% formulae
\usepackage{amssymb,amsmath}
\usepackage{ifxetex,ifluatex}
\usepackage{fixltx2e}
\usepackage[T1]{fontenc}
\usepackage[utf8]{inputenc}

% hyperlinks
\IfFileExists{upquote.sty}{\usepackage{upquote}}{}
\IfFileExists{microtype.sty}{%
\usepackage[]{microtype}
\UseMicrotypeSet[protrusion]{basicmath} % disable protrusion for tt fonts
}{}
\PassOptionsToPackage{hyphens}{url} % url is loaded by hyperref
\usepackage[unicode=true]{hyperref}
  \PassOptionsToPackage{usenames,dvipsnames}{color} % color is loaded by hyperref
\definecolor{maroon}{cmyk}{0, 0.87, 0.68, 0.32}
\hypersetup{
      colorlinks=true,
    linkcolor=maroon,
    citecolor=blue,
    urlcolor=blue,
      breaklinks=true}
\urlstyle{same} % don't use monospace font for urls

% geometry
\usepackage[left=2cm,right=2cm,top=2cm,bottom=2cm]{geometry}
\renewcommand{\baselinestretch}{1.1}

% Bibliography
\usepackage{natbib}
\bibliographystyle{plainnat}

% Tables
\usepackage{longtable,booktabs}
% Fix footnotes in tables (requires footnote package)
\IfFileExists{footnote.sty}{\usepackage{footnote}\makesavenoteenv{long table}}{}

% graphics
\usepackage{graphicx,grffile}
\makeatletter
\def\maxwidth{\ifdim\Gin@nat@width>\linewidth\linewidth\else\Gin@nat@width\fi}
\def\maxheight{\ifdim\Gin@nat@height>\textheight\textheight\else\Gin@nat@height\fi}
\makeatother
\setkeys{Gin}{width=\maxwidth,height=\maxheight,keepaspectratio}

% indent
\IfFileExists{parskip.sty}{%
\usepackage{parskip}
}{% else
\setlength{\parindent}{0pt}
\setlength{\parskip}{6pt plus 2pt minus 1pt}
}

% prevent overfull lines
\setlength{\emergencystretch}{3em}  
\providecommand{\tightlist}{%
\setlength{\itemsep}{0pt}\setlength{\parskip}{0pt}}

\setcounter{secnumdepth}{0}

% paragraphs
\usepackage{titlesec}
\let\oldsection\section
\renewcommand\section{\newpage\oldsection}
\titleformat{\section}
{\huge\center\scshape}{\thesection}{1em}{}[{\titlerule[0.8pt]}]
\titleformat*{\subsection}{\LARGE\bfseries}
\titleformat*{\subsubsection}{\Large\bfseries}
\titleformat*{\paragraph}{\large\bfseries\itshape}
\titleformat*{\subparagraph}{\normalsize\itshape}

% citations
\usepackage{epigraph}

% set default figure placement to htbp
\makeatletter 
\def\fps@figure{htbp}
\makeatother

%%% BODY %%%
\usepackage{amsthm}
\newtheorem{theorem}{Theorem}[section]
\newtheorem{lemma}{Lemma}[section]
\theoremstyle{definition}
\newtheorem{definition}{Definition}[section]
\newtheorem{corollary}{Corollary}[section]
\newtheorem{proposition}{Proposition}[section]
\theoremstyle{definition}
\newtheorem{example}{Example}[section]
\theoremstyle{remark}
\newtheorem*{remark}{Remark}
\begin{document}

% First pages
  % First page
  \includegraphics{images/logos}
  
  \begin{center}
    \vspace*{\stretch{1}}
    \LARGE{\textbf{Mémoire de stage}} \\
    \vspace*{\fill}
    \large{présenté par} \\
    \large{Sylvain SCHMITT} \\
    \vspace*{\fill}
    \large{pour obtenir le diplôme national de master} \\
    \large{mention Biodiversité, écologie, évolution} \\
    \small{parcours Biodiversité végétale et gestion des écosystèmes tropicaux (BIOGET)} \\
    \vspace*{\fill}
    \large{Sujet :} \\
    \Large{\textbf{Rôle de la biodiversité dans la résilience des écosystèmes forestiers tropicaux après perturbation}} \\
    \vspace*{\fill}
    \large{soutenu publiquement le XX Juin 2017} \\
    \large{à Kourou} \\
    \vspace*{\fill}
    \large{devant le jury suivant :} \\
    \vspace*{\fill}
    Dr Bruno HÉRAULT  \emph{Tuteur de stage} \\
    Titre Prénom NOM  \emph{Examinateur} \\
    Titre Prénom NOM  \emph{Examinateur} \\
    Dr Stéphane TRAISSAC  \emph{Enseignant-réferent} \\
    \vspace*{\stretch{1}}
  \end{center}
  
  % Second page
  \newpage
  \vspace*{\fill}
  \epigraph{Le simple est toujours faux. Ce qui ne l'est pas est inutilisable.}{\textit{Paul Valéry}}
  \vspace*{\fill}
  \emph{Les opinions émises par les auteurs sont personnelles et n'engagent pas AgroParisTech.}\\
  \emph{Une version web de ce document est diponible à l'adresse suivante : \url{https://sylvainschmitt.github.io/master-thesis/}.}
  \newpage

\tableofcontents

\section*{Résumé et Abstract}\label{resume-et-abstract}
\addcontentsline{toc}{section}{Résumé et Abstract}

Les forêts tropicales font face à de nombreuses perturbations qui
représentent la troisième source mondiale d'émission de gaz à effet de
serre. La déforestation et la dégradation des forêts tropicales sont
responsables de l'émission de 8.26 milliards de tonnes de dioxyde de
carbone par an \citep{Pearson2017}. La déforestation a retenu
l'attention mondiale, mais la dégradation des forêts représente 20\% des
émissions de l'Amazonie brésilienne \citep{Asner2005}. La gestion
durable des forêts a été proposée comme réponse à la déforestation et la
dégradation, malgré la remise en question de la durabilité de
l'exploitation forestière \citep{Zimmerman2012}. D'autre part, les
forêts tropicales abritent plus de la moitié de la biodiversité
terrestre mondiale \citep{Scheffers2012}. Par conséquent, nous avons
décidé d'étudier le rôle de la biodiversité dans la réponse des
écosystèmes forestiers aux perturbations, en reliant diversité et
fonctionnement de l'écosystème \citep{Loreau2010}. Nous avons utilisé
l'hypothèse que lors d'une perturbation, grâce à une productivité plus
forte, une forêt plus diverse aura une meilleure résilience, en se
basant sur la relation positive entre biodiversité et productivité. Nous
avons relié cette hypothèse aux effets de complémentarité et de
sélection \citep{Loreau2001}. La complémentarité est la combinaison de
la partition des ressources et de la facilitation, alors que l'effet de
sélection est le résultat de la sélection compétitive. Nous avons ainsi
centré l'étude sur les mécanismes impliqués dans la relation entre
biodiversité et résilience des écosystème forestiers par une approche
par simulation afin d'appréhender les processus à long terme. Nous avons
utilisé le modèle TROLL \citep{Li} pour simuler 60 forêts matures aux
diversités taxonomiques et fonctionnelles croissantes. Nous avons
perturbé toutes les forêts et mesuré la résilience de leurs fonctions
écosystémiques. En outre, nous avons mesuré la résilience de l'effet net
de la biodiversité que l'on a décomposé en en effets de complémentarité
et de sélection. Nous avons trouvé que la diversité améliore la
résilience des forêts tropicales, particulièrement au travers de la
diversité et l'équitabilité fonctionnelle. De plus, nous avons montré
que la complémentarité entre les espèces assurait la résilience de la
forêt en début de succession avant de laisser place à l'effet de
sélection. Nos résultats suggèrent la possibilité d'une gestion durable
des forêts tropicales grâce à une meilleure résilience avec une plus
haute diversité. Mais cette conclusion n'a de sens que si l'exploitation
sélective est durable \citep{Zimmerman2012}. Au contraire, une gestion
non durable des forêts tropicales entraînera des rétroactions négatives
diminuant lentement la diversité et donc la résilience des forêts,
aboutissant ultimement à la dégradation des forêts.

Forest disturbances are the third worldwide source of greenhous gas.
Tropical deforestation and degradation emit 8.26 billion of tons of
carbon dioxyde per year \citep{Pearson2017}. Deforestation has retained
much attention, but degradation from forest represents 20\% of emissions
in brazilian Amazon \citep{Asner2005}. Sustainable forest management has
been promoted as an answer to deforestation and degradation, besides
logging sustainability has been questionned \citep{Zimmerman2012}. On
the other hand, tropical forest host over half of the Earth's
biodiversity \citep{Scheffers2012}. Consequently, we decided to study
the role of biodiversity in forest ecosystem answer to disturbance,
linking diversity to ecosystem functionning \citep{Loreau2010}. We used
the hypothesis that when a disturbance event happen, due to a higher
productivity, a more diverse forest will be more resilient, based on the
positive relationship between biodiversity and productivity. We linked
that hypothesis to the complementarity and selection effects
\citep{Loreau2001}. Complementarity is the addition of ressource
partitionning and facilitation, whereas selection effect is the result
of competitive selection. We thus focused on mechanisms involved in the
relationship between biodiversity and forest ecosystem resilience with a
simulation approach to assess long term processes. We used TROLL model
\citep{Li} to simulate 60 matures forests with growing taxonomic and
functional diversities. We disturbed all forests and measured the
resilience of their ecosystem functions. Additionnally, we measured
biodiversity net effect resilience partitioned into complementarity and
selection effects. We found that diversity improved tropical forest
resilience, particularly through functional diversity and eveness.
Moreover, we showed that complementarity between species insured forest
recovery in the beginning of the succession before being replaced by
selection effect. Our results suggest the possibility for a sustainable
management of tropical forest due to an increased resilience with an
higher diversity. But this conclusion has meaning only if selective
logging meet sustainability \citep{Zimmerman2012}. On the contrary,
unsustainable tropical forest management will lead to negative feedbacks
slowly diminishing diversity and thus forest resilience, resulting
ultimately in forest degradation.

\section*{Acknowledgments}\label{acknowledgments}
\addcontentsline{toc}{section}{Acknowledgments}

I would like to thank Bruno Hérault and Stéphane Traissac for their
guidance and advice during the whole internship. I congratulate Bruno to
have been able to communicate and be reactive besides his numerous
recent travels. I am grateful to Bruno and Camille Piponiot to have
introduced me to the wonderful world of Bayesian statistics. All this
work would not have been possible without TROLL model and the whole team
that works on it : Jérôme Chave, Isabelle Maréchaux and Fabian Fischer.
I thank all of the team to have introduced me to the model and guide me
in its development from the opposite side of the Atlantic. I would also
like to thank Laurent Descroix and people from National Forest Office
who introduced me to sylviculture from french Guiana and helped me to
build a realistic forest model. I also thank Aurélie Dourdain for her
help to access Paracou data, Éric Marcon for his advice on biodiversity
analysis, and Pascal Padolus for his patience and help wiht Tabebuia
cluster. The study would not have been possible without Pascal
Petronelli and all people who have contributed to Paracou and Guyafor
data collection over the years. Finally I am grateful to Maxime
Réjou-Méchain, Raphaël Pélissier, and AMAP colleagues, in addition to
Myriam Heuertz and BIOGECO colleagues and Aurélie Cuvélier who helped me
discuss the study results and build the present master thesis. To
conclude, I thank everybody that played a role more or less important in
the development of my internship and that I might have forgotten.

\section*{Introduction}\label{introduction}
\addcontentsline{toc}{section}{Introduction}

Tropical forests disturbances, through deforestation and degradation,
account for 8.26 billion of tons of carbon dioxyde emissions per year
\citep{Pearson2017}. It represents the third source of greenhouse gas.
Besides deforestation has been rightly the focus of worldwide attention,
degradation has been less studied and quantified through tropics.
Degradation from forest has been estimated to represents 10 time
deforestation \citep{Herold2011} and represents 20\% of greenhouse gas
emissions in the brazilian Amazon \citep{Asner2005}. Degradation has met
numerous definitions \citep{Simula2009}, but can be defined as the
result from a disturbance event that induced a modification of the
forest ecosystem reducing ecosystem services while keeping the ecosystem
as a forest, contrary to deforestation.

Sutainable forest management in the tropics (i.e.~managed selective
harvesting of timber) has been widely promoted internationnaly to combat
tropical deforestation and degradation \citep{Zimmerman2012}. Currently
logging from tropical forests accounts for one eight of global timber
production \citep{Blaser2011} and is still increasing. Most tropical
timber production originates from selective logging, the targeted
harvesting of timber from commercial species in a single cutting cycle
\citep{Martin2015}.

On the other hand, tropical rainforests have fascinated ecologists due
to their outstanding diversity \citep{connell_diversity_1978}.
Effectively tropical forests host over half of the Earth's biodiversity
\citep{Scheffers2012}. High biodiversity from tropical rainforests is
the source of many ecosystem functions. Amongst others, tropical forests
play a key role in biogeochemical cycles, including carbone storage
\citep{Lewis2004}. Ecosystem functions from tropical forests support
numerous ecosystem services, such as timber production and climate
regulation.

But several authors argue that selective logging represents a major
threat to biodiversity
\citep{Carreno-Rocabado2012, DeAvila2015, Gibson2013, Martin2015, Zimmerman2012},
challenging the sustainable definition from current selective logging.
We consequently need to assess both short and long term impacts of
selective logging on tropical forest ecosystems to implement better
sylvicultural practices in order to reach sustainability.

The question of selective logging impact on tropical forest can be
directly related to the emerging field of biodiversity and ecosystem
functionning \citep{Loreau2000}. Tropical forest outstanding
biodiversity will be both a factor and a result of forest ecosystem
response to logging disturbance. And forest ecosystem response to
logging disturbance will directly modify ecosystem functionning in both
short and long term. Consequently assessing selective logging effect on
tropical forest linking diversity and ecosystem functionning seems an
obvious and promising way \citep{Loreau2010}.

Negative short term impacts of selective logging have been assessed
\citetext{\citealp{Carreno-Rocabado2012}; \citealp{DeAvila2015}; \citealp[but
see][]{Martin2015}}. Much less is known about the long term impact
\citep{Osazuwa-Peters2015}. The main reason is the difficulty to conduct
long term empirical study \citep[but see][]{Herault2010}, which can be
completed by the use of forest simulators
\citep{Huth2004, Khler2004, Ruger2008, Tietjen2006}. Individual-based
models of forest dynamics present the perfect framework to develop such
joint biodiversity-ecosystem approaches \citep{Li}. Individual-based
models describe forest accumulating carbon through time, assessing tree
growth, or releasing carbon through gap opening \citep{Bugmann2001}. Up
to several dozens of different Plant Functional Types (PFTs) are
generally defined and models can sometimes be fully spatially explicit
\citep{Pacala1996}. Recently, the forest growth simulator TROLL
\citep{Chave1999}, an individual-based and spatially explicit forest
model, was developped to introduce recent advances in plant
physiological community. TROLL model relates physiological processes to
species-specific functional traits \citep{Li}. Consequently, TROLL model
allow to simulate fully a neotropical forest biodiversity to study
biodiversity-ecosystem functionning link response to logging
disturbance.

We decided to use the forest model TROLL to study the role of
biodiversity in forest ecosystem answer to disturbance. Resilience
encompass several definitions but was summarized by \citet{Oliver2015}
as the degree of resistance or fast recovery from an ecosystem function
to environmental disturbance.

Our work is based on the general hypothesis of a positive relationships
between biodiversity and productivity. We assumed that when a
disturbance event happen, due to a higher productivity, forest with an
increased diversity will recover quicker and thus be more resilient.
This theoritical expectation stand on two processes: complementarity and
selection effects \citep{Loreau2001}. If we take a species pool with
different productivity in monoculture. Their assemblage will allow theim
an overall higher productivity due to a better ressources acquisition
through ressources partitionning and niche differentiation.
Additionnally, some species will individually have an higher
productivity in the assemblage than in monoculture due to facilitation
from other species present in the assemblage. Complementarity effect is
the addition of the better ressource acquisition and the facilitation.
Now if we look at the evolution of the species assemblage over time,
more competitive species will progressively dominate through competitive
selection. And if competitive species are more productive, they will
increase assemblage overall productivity. This ressources preemption by
more competitive species is the selection effect. Moreover, we expect
the sampling effect to improve more diverse forest. A bigger initial
sampling of the regional species pool in a rich forest, will allow more
redundancy and a lower risk to lose important functional traits
assemblages for the ecosystem when the disturbance happen.

Generally, positive relationship between biodiversity and productivity
were emprically and experimentally demonstrated on grassland systems
\citep{Hooper2005a, Loreau2001a, Naeem2002}; but few studies focused on
the case of tropical forests. One of the few studies was realized by
\citet{Chisholm2013} and has shown a positive significative relationship
between species richness and wood productivity on a worlwide forest
network. But those study still presents three major limits: (1) study
time are inferior to a tree life time, (2) experimental network include
scarcely disturbed plot, and (3) correlative approach does not explain
mechanisms involved in the relationship.

In the present study, we focused on mechanisms involved in the
relationship between biodiversity and forest ecosystem resilience. We
used a simulation approach using TROLL model to assess long term
processes for different types and levels of disturbances. We first
assessed diversity effect on forest resilience of structure and
functionning using numerous indices of taxonomic and functional
diversities. Then, we measured biodiversity net effect, partitioned into
complementarity and selection effects, resilience for several ecosystem
metrics.

\section{Model description}\label{model-description}

\subsection{Overview}\label{overview}

TROLL model each tree indivdually in a located environment. Thus TROLL
model, alongside with SORTIE \citep{Pacala1996, Uriarte2009} and FORMIND
\citep{Fischer2016, Kohler1998}, can be defined as an individual-based
and spatially explicit forest growth model. TROLL simulates the life
cycle of individual trees from recruitment, with a diameter at breast
height (dbh) above 1 cm, to death with growth and seed production. Trees
are growing in a spatialized light environment explicitly computed witin
voxels of 1 \(m^3\). Each tree is consistently defined by its age,
diameter at brest height (dbh), height (h), crown radius (CR), crown
depth (CD) and leaf area (LA) (see figure \ref{fig:TROLLtree}). Tree
geometry is calculated with allometric equations but leaf area vary
dynamically within each crown following carbon allocations. Voxels
resolution of 1 \(m^3\) allow the establishment of maximum one tree by
1x1 m pixels. Each tree is flagged with a species label inherited from
the parent tree through the seedling recruitment. A species label is
associated to a number of species specific parameters (see table
\ref{tab:traits}) related to functional trait values which can be
sampled on the field.

\begin{figure}[htbp]
\centering
\includegraphics{images/TROLLtree.png}
\caption{\label{fig:TROLLtree}Individuals tree inside TROLL explicit spatial
grid from \citet{Li}. Tree geometry (crown radius CR, crown depth CD,
height h, diameter at breast height dbh) is updated at each timestep
following allometric relationship with assimilated carbon allocated to
growth. Each tree is flagged with a species label linking to its
species-specific attributes. Light is computed explicitly at each
timestep for each voxel.}
\end{figure}

Carbon assimilation is computed over half-hourly period of a
representative day. Then allocation is computed to simulate tree growth
from an explicit carbone balance (in contrast to previous models).
Finally environment is updated at each timestep set to one month.
Seedlings are not simulated explicitly but as a pool. In addition
belowground processes, herbaceous plants, epiphytes and lianas are not
simulated inside TROLL. The source code is written in C++ and available
upon request. All modules of TROLL models are further detailed in
\protect\hyperlink{appendix-1-troll-model}{Appendix 1: TROLL model}.

\begin{longtable}[]{@{}lll@{}}
\caption{\label{tab:traits}Species-specific parameters used in TROLL from
\citet{Li}. Data originates from the BRIDGE \citep{Baraloto2010} and TRY
\citep{Kattge2011} datasets.}\tabularnewline
\toprule
Abbreviation & Description & Units\tabularnewline
\midrule
\endfirsthead
\toprule
Abbreviation & Description & Units\tabularnewline
\midrule
\endhead
\(LMA\) & leaf mass per area & \(g.m^{-2}\)\tabularnewline
\(N_m\) & leaf nitrogen content per dry mass &
\(mg.g^{-1}\)\tabularnewline
\(P_m\) & leaf phosphorous content per dry mass &
\(mg.g^{-1}\)\tabularnewline
\(wsg\) & wood specific gravity & \(g.cm^{-3}\)\tabularnewline
\(dbh_{thresh}\) & diameter at breasth height threshold &
\(m\)\tabularnewline
\(h_{lim}\) & asymptotic height & \(m\)\tabularnewline
\(a_h\) & parameter of the tree-height-dbh allometry &
\(m\)\tabularnewline
\bottomrule
\end{longtable}

Previous implementation of TROLL model used \citet{Reich1991a} allometry
to infer leaf lifespan \(LL\) from species leaf mass per area \(LMA\)
\citep[see \protect\hyperlink{appendix-1-troll-model}{Appendix 1: TROLL
model}]{Li}. But the use of the allometrie from \citet{Reich1991a} with
current implementation of the TROLL model resulted in an underestimation
of leaf lifespan for low LMA species. Consequently in the following
paragraph we suggest a new allometry.

Selective logging is defined as the targeted harvesting of timber from
species of interest. Consequently, tropical sylviculture can be
assimilated to a disturbance. The main difference between a disturbance
and selective logging is the targetting of both species and individuals
of interest. So we decided to first asses unselective disturbance effect
on tropical forest ecosystem to subsequently better understand selective
logging effect. First, we implemented a disturbance module inside TROLL
model to simulate unselective disturbance. Secondly, we implemented a
sylviculture module inside TROLL model to simulate selective logging in
regards to french Guiana practices.

\subsection{Leaf lifespan}\label{leaf-lifespan}

The underestimation of leaf lifespan for low LMA species with the
allometry from \citet{Reich1991a} resulted in indivduals unealistic
early death from carbon starvation. We gathered data from
TRY\citep{Kattge2011}, DRYAD \citep{chave_towards_2009} and GLOPNET
\citep{wright_worldwide_2004} datasets. We used an out of the bag method
applied on a random forest to select variables with highest importance
to explain leaf lifespan. We thus selected leaf mass per area \(LMA\),
leaf nitrogen content \(N\) and wood specific gravity \(wsg\). We then
used a bayesian approach to test different models with growing level of
complexity. The model with the best tradeoff between complexity (number
of parameters), convergence, likelihood, and prediction quality (root
mean square error of prediction RMSEP) was kept. We selected following
model with a maximum likelihood of 13.6 and a RMSEP of 12 months:

\begin{equation}
  LL_{d} \sim log\mathcal{N}({\beta_1}_d*LMA - {\beta_2}_d*N*\beta_3*wsg, \sigma)
  \label{eq:LL}
\end{equation}

Leaf lifespan \(LL\) follows a lognormal law with location infered from
leaf lifespan \(LMA\), nitrogen content \(N\) and wood specific gravity
\(wsg\) and a scale \(\sigma\). Each \({\beta_i}_d\) is following a
normal law located on \(\beta_i\) with a scale of \(\sigma_i\). All
\(\beta_i\), \(\sigma_i\), and \(\sigma\) are assumed without
presemption following a gamma law. \(d\) represents the dataset random
effects and encompass environmental and protocol variations (see
\protect\hyperlink{appendix-2-leaf-lifespan-model}{Appendix 2: Leaf
lifespan model} for more details).

\subsection{Disturbance}\label{disturbance}

Disturbance module was designed in the simplest way in order to relate
the ecosystem answer to volume loss without any individuals nor species
targetting. For a given iteration \(disturb_{iter}\), individuals are
picked randomly with a uniform law on the number of trees. Selected
individuals are then removed without trigerring a treefall to avoid any
side effect. The operation is repeated untill the disturbance result in
a defined lost basal area (\(disturb_{intensity}\) in \% of BA).

\subsection{Sylviculture}\label{sylviculture}

In french guiana context, sylviculture can be narrow to selective
logging, which can be split in two steps: selection and harvesting.
Selection encompass choice of the havrestable area, harvestable tree
designation by the forest office, harvested tree selection by the
harvester, and removal off tree probbed as rotten by the lumber.
Harvesting encompass tree felling, tracks opening, and long term damages
(simplified in gap damages in current TROLL implementation).

\subsubsection{Designation and
selection}\label{designation-and-selection}

One major limit of current implementation of TROLL model is that it
assumes a flat environment. Consequently the whole simulated area inside
TROLL is considered has an harvestable zone. With all commercial species
minimum and maximum harvestable diameter, TROLL calculates the total
harvestable volume \({V_h}_{tot}\). If the total harvestable volume
\({V_h}_{tot}\) exceed \(30~m^3.hectare^{-1}\), commercial species
minimum harvestable diameter \(dbh_{min}\) is increased untill
\({V_h}_{tot}\) is inferior to that upper limit.

In french guiana, tree harvesters are focusing on few species with
easier marketable wood, resulting in a tree harvest around
\(20~m^3.ha^{-1}\) (Laurent Descroix, ONF, personnal communication).
TROLL ranks each commercial species on its economic value, and randomly
remove individuals from lowest rank species untill it reaches total
harvested volume \({V_{hd}}_{tot}\) (\({V_{hd}}_{tot}\) was set to
\(25~m^3.hectare^{-1}\) in subsequent simulations).

\subsubsection{Rotten trees}\label{rotten-trees}

20 to 30 \% of designated trees are considered as rotten once probed by
the lumberman, and thus not harvested. Rotten trees are not random and
depends both on tree species and diameter. We gathered data from the
forest office (ONF, Laurent Descroix, personnal communication)
inventories precising if tree were probbed as rotten and their
corresponding species and diameter. In addition, tree plots and sawed
volume was informed. We then used a bayesian approach to model the link
between tree species and diameter and their risk to be probbed as rotten
by the lumberman. We test different models with growing level of
complexity and kept the model with the best tradeoff between complexity
(number of parameters), convergence, likelihood, and prediction quality
(root mean square error of prediction RMSEP):

\begin{equation}
  \begin{array}{c} 
    probbed~rotten \sim \mathcal{B}(P(probbed~rotten)) \\
    P(probbed~rotten) = logit^{-1}(\beta_0 + \beta_1*dbh) = \frac{e^{\beta_0 + \beta_1*dbh}}{1 + e^{\beta_0 + \beta_1*dbh}}
  \end{array}
  \label{eq:rotten}
\end{equation}

Tree \(probbed~rotten\) follows a \(Bernoulli\) law of probability
\(P(probbed~rotten)\). The odds for a tree to be probbed as rotten are
calculated with the sum of a base odd to be rotten \(\beta_0\) and a
diameter dependent odd calculated with \(\beta_1\). The probability for
a tree to be probbed as rotten \(P(probbed~rotten)\) is finally
calculated by taking the inverse logit \(logit^{-1}\) of the odd (see
\protect\hyperlink{appendix-3-rotten-tree-model}{Appendix 3: Rotten tree
model} for more details).

\subsubsection{Harvesting}\label{harvesting}

Due to crown aspects, treefall from logs are often random (whereas
difficult to manage, treefall can still be oriented, Laurent Descroix,
ONF, personnal communication). Consequently, TROLL consider treefall
from log as random like current natural treefall implementation inside
TROLL (see code {[}Appendix 1{]}).

Tree harvesting roads are split in three classes: truck roads, main
tractor track, and secondary track. Because TROLL assumes a flat
environment, the main track is opened starting from the midle of one
side of the simulated forest and untill it reaches the center with a
width of 6 meters. In most cases, secondary tracks are opened once trees
have been designated and the geolocation taken at a maximum distance of
30 meters from designated trees (Laurent Descroix, ONF, personnal
communication). To simulate secondary tracks, TROLL uses a loads map,
measuring every trees at a distance of 30 meters for each pixel, and a
track proximity maps of the closest existing track. Next, the model
select the pixel with the highest load and closest track, find the
closest existing track and join it by removing tree in the way with a
width of 5 meters. The operations are repeated untill no felt trees are
left.

\subsubsection{Gap damages}\label{gap-damages}

Most of models account long term damages due to selective logging with a
10 years increased mortality \citep{Huth2004, Khler2004, Ruger2008}. We
decided to model explicitly long term logging damages because of their
localised nature through a gap damages model. We gathered data from
Paracou dataset \citep{Guehl2004} in cenususes between 1988 and 1992 on
Paracou harvested plots. Individuals were categorized between alive,
dead, or recruited during the period. We measured each individual
distance to the closest gap. We then used a bayesian approach to test
the link between tree death in the four years following the log event
and distance to the closest gap. We adapted the model from
\citet{Herault2010} based on a disturbance index into:

\begin{equation}
  \begin{array}{c} 
    Death \sim \mathcal{B}(P(Death)) \\
    P(Death) = logit^{-1}(\theta + \beta*e^{\alpha*d_{gaps}}) = \frac{e^{\theta + \beta*e^{\alpha*d_{gaps}}}}{1 + e^{\theta + \beta*e^{\alpha*d_{gaps}}}}
  \end{array}
  \label{eq:death}
\end{equation}

\(Death\) of a tree follows a \(\mathcal{B}ernoulli\) law of probability
\(P(Death)\). The odds for a tree to die are calculated with the sum of
the natural tree death odd \(\theta\) and a perturbation index
\(\beta*e^{\alpha*d_{gaps}}\). The perturbation index depend on the
distance \(d_{gaps}\) of the tree \(i\) to the closest logging gap. The
probability for a tree to die \(P(Death)\) is finally calculated by
taking the inverse logit \(logit^{-1}\) of the odd.

\section{Material and Methods}\label{material-and-methods}

\subsection{Sensitivity analysis}\label{sensitivity-analysis}

\citet{Li} already assessed TROLL model sensitivity to several
parameters (\(k\) see \eqref{eq:PPFD}, \(\phi\) see \eqref{eq:LCP}, \(g1\)
see \eqref{eq:gs}, \(f_{wood}\) see \eqref{eq:DeltaV}, \(f_{canopy}\) see
\eqref{eq:DeltaLA} and \(m\) see \eqref{eq:db}) which they assumed having a
key role in model functioning. On the other hand, we decided to use
TROLL to study resistance and resilience of ecosystem face to
disturbance, highlighting the role of biodiversity. Consequently we
particulrarly needed to assess the importance of functional traits to
further better control and evaluate functional diversities. We also
needed to assess the sensitivity of TROLL model to the seed rain
constant (\(n_{ext}\), see \eqref{eq:next}) because we assumed it was one
of the main factors of tree recruitments after disturbance within
simulations.

TROLL model currenty uses leaf mass per area (\(LMA\) in \(g.m^{-2}\)),
leaf nitrogen content per dry mass (\(N_m\) in \(mg.g^{-1}\)), leaf
phosphorus content per dry mass (\(P_m\) in \(mg.g^{-1}\)), wood
specific gravity (\(wsg\) in \(g.cm^{-3}\)), diameter at breasth height
threshold (\(dbh_{thresh}\) in \(m\)), asymptotic height (\(h_{lim}\) in
\(m\)), and parameter of the tree-height-dbh allometry (\(a_h\) in
\(m\)). To assess the sensitivity of TROLL model to species functionnal
traits, we performed a sensitivity analysis by fixing species trait
values to their mean. Each trait was tested independently. We reduce to
a common mean traits with a Pearson's correlation value \(r \geq 0.8\)
(\(h_{max}\) and \(a_h\) with a correlation of \(r=0.98\)). To assess
the sensitivity of TROLL model to seed rain, we performed a sensitivity
analysis by fixing simulations seed rain constant to 2, 20, 200 and 2000
seeds per hectare.

Simulations were conducted on Intel Xeon(R) with 32 CPUs of 2.00GHz and
188.9 GB of memory. We assumed maturity of the forest after 500 years of
regeneration \citet{Li} and computed simulation 100 years after a
disturbance event with 40\% loss of basal area. Due to computer
limitations we did not run replicates (besides it should be necessary to
reduce simulation stochasticity). To assess ecosystem outputs
sensitivity to studied parameters, we compared it to 100 replicates of
control simulations with all parameters set to default values. Ecosystem
outputs outside of the range of the control replicates values are
significantly influenced by the studied parameter.

\subsection{Design of experiment}\label{design-of-experiment}

In order to assess the role of biodiversity in ecosystem answer to both
disturbance and sylviculture, we needed to create a space of experiments
encompassing both variation of disturbance, biodiversity and time.
Disturbance was represented by percentage of basal area loss (0\%, 25\%,
50\% and 75\%), or as a selective logging simulation using default
parameters. Biodiversity was integrated with two of its components:
taxonomic and functional diversities. We used species richness \(SR\) to
represents taxonomic diversity (5, 25, and 125 species). Functional
diversity can be related to numerous components, and \citet{Borgy2017}
argued for 5: richness, divergence, regularity, overlap and mean.
Because mature forest were created from a bare soil with TROLL
simulations, we could not control a priori divergence, regularity and
overlap but only assess them after running the simulations, i.e.~before
applying the diturbance. Consequently, we focused on functional richness
with convex hull volume \(CHV\) and functional mean with community
weighted mean \(CWM\). For each level of species richness \(SR\), we
selected 20 communities with growing convex hull volume \(CHV\) but with
a community weighted means close to the regional species pool community
weighted means. Effectivelly, we did not wanted drastic change in
community means that could have more effect than functional richness
itself. This design of experiments resulted in 60 communities
(\(5~SR*20~CHV\)) and 240 simulations
(\(60 ~communities*4~levels~of~disturbance\)) over 600 years (maturity
being assumed after 500 years of regeneration \citep{Li}). Functional
diversities of mature forests were assessed with
\citet{villeger_new_2008} indices (FRIC, FEve, FDiv, and FDis). Figure
\ref{fig:DOE} presents the design of experiment for communities
biodiversity after the mature forest were simulated, and thus before
disturbance. We obtained a broad range of both functionl dispersion
\(FDis\) and aboveground biomass \(AGB\) for simulated forest ecosystems
before disturbance.

\begin{figure}[htbp]
\centering
\includegraphics{master-thesis_files/figure-latex/DOE-1.pdf}
\caption{\label{fig:DOE}Experimental design before disturbance. Communities
are implemented along a gradient of species richness (SR) and functional
dispersion (FDis) resulting in a broad range of aboveground biomass
(AGB). FDis was caluclated based on 4 functional traits (leaf mass per
area, wood specific gravity, maximum diameter, and maximum height).}
\end{figure}

\subsection{Ecosystem response
analysis}\label{ecosystem-response-analysis}

\subsubsection{Ecosystem functions}\label{ecosystem-functions}

Tropical forest ecosystems provides numerous ecosystem services linked
to several ecosystem functions. We decided to describe simulated
tropical forests in two major functions: forest structure and forest
functionning. Forest structure was represented by aboveground biomass
(\(AGB\) in \(ton~C.ha^{-1}\)), basal area (\(BA\) in \(m^2.ha^{-1}\)),
total number of stem (\(N\)), number of stem above \(10~cm\) diameter
(\(N10\)), and number of stem above \(30~cm\) diameter (\(N30\)). Forest
functionning was represented by growth primary productivity (\(GPP\) in
\(MgC.ha^{-1}\)), net primary productivity (\(NPP\) in \(MgC.ha^{-1}\)),
tree autotrophic respiration in day (\(Rday\) in \(MgC.ha^{-1}\)) and
tree autotrophic respiration in night (\(Rnight\) in \(MgC.ha^{-1}\)).

The resilience of metrics values post disturbance were assessed through
\citet{Henry2012} formula:

\begin{equation}
  R\left(t\right)=\frac{Recovery\left(t\right)}{Loss\left(t_d\right)} \approx \frac{X_T(t)}{X_C(t)}
  \label{eq:Resilience}
\end{equation}

The resilience of the system \(R(t)\) at the time \(t\) is described by
the ratio of recovery \(Recovery(t)\) at time \(t\) to loss suffered
\(Loss(t_d)\) at disturbance time \(t_d\). But in our peculiar case of
tropical forest ecosystems, the equilibrium used to calculate
\(Loss(t_d)\) can not be reduced to a specific time if the equilibrium
is dynamic. Consequently, to encompass undisturbed ecosystem variations
throught time, we simulated an undisturbed control ecosystem \(C\). And
the resilience of the system \(R(t)\) at the time \(t\) was defined as
the ratio of the ecosystem metric values in the disturbed simulation
\(X_T(t)\) over the ecosystem metric values from the control \(X_C(t)\)
. Thus, the value of resilience \(R(t)\) is normalized for all
simulations and metrics. \(R(t)\) will be equal to \(R_{eq} = 1\) when
reaching the equilibrium value. Consequently we can calculate an
euclidean distance to equilibrium \(d_{eq}(t)\) as
\(d_{eq}(t) = \sqrt{(R_{eq} - R(t))^2}\). Ecosystem euclidean distance
to equilibrium was calculated in a multi-dimensional space for the two
functions described above: forest strcuture (AGB, BA, N, N10, and N30)
and forest functionning (GPP, NPP, Rday, and Rnight). We then used
integrated eulcidean distance to equilibrium over time to assess
simulations resilience.

\subsubsection{Biodiversity effect}\label{biodiversity-effect}

Biodiversity is not only a facet of the experimental design and an
ecosystem output through forest diversity, but also interact on
ecosystem functioning and consequently on its answer to disturbance.
Biodiversity ecosystem functioning relation can be split in
complementarity and selection effect with \citet{Loreau2001}
partitioning:

\begin{equation}
  \begin{array}{c}
    NE = X_O - X_E = CE + SE \\
    CE = N* \overline{\Delta RX} \overline{M}\\
    SE = N*cov(\Delta RX,M)
  \end{array}
  \label{eq:BiodivPart}
\end{equation}

Biodiversity net effect \(NE\) is based on the difference between
ecosystem variable \(X\) observed value \(X_O\) within the community
mixture of species and its expected value \(X_E\) if species performance
were equal to their performance in monocultures. This effect can be
partitioned between complementarity effect \(CE\), representing niche
partitionning, positive interactions, and resource supply, and selectvie
effect \(SE\) due to dominant species pool driving the ecosystem. \(N\)
represents the total number of species, and \(M\) the vector of
monocultures performance. Both metrics depend on the variation of
relative ecosystem variable \(\Delta RX\):

\begin{equation}
  \Delta RX_{sp} = \frac{X_{sp}(mixture)}{X_{sp}(monoculture)} - P_{sp}
  \label{eq:DeltaRY}
\end{equation}

\(X_{sp}\) is the ecosystem variable value for one species either in
mixture \(X_{sp}(mixture)\) or in monoculture \(X_{sp}(monoculture)\).
\(P_{sp}\) is the proportion of the species in the mixture represented
by species relative abundance. Consequently, \(CE\) averages diversity
effects of all species presents in the mixture (both negatives and
positives). Whereas \(SE\) become positive when dominant species
outperform themselves in mixture than in monoculture, and negative when
less dominant species outperform themselves in mixture than in
monoculture \citep{Tobner2016}. But similarly to resilience measurement,
biodiversity net effect \(NE\) is a dynamic equilibrium and vary over
time without disturbance. So in order to correctly assess selection and
complementarity effect in answer to disturbance, we normalized it by
undisturbed control ecosystem net effect \(NE_C\) to measure treatments
net effect resilience \(R(NE_T)\):

\begin{equation}
  R(NE_T) = \frac{NE_T}{NE_C} = \frac{SE_T}{NE_C} + \frac{CE_T}{NE_C}
  \label{eq:RNE}
\end{equation}

Resilience trajectories of ecosystem variable after disturbance were
partitioned between complementarity effect \(CE\) and selection effect
\(SE\). In order to do that, the design of experiment was repeated for
each species indivdually representing 652 simulations of monoculture.

\section{Results}\label{results}

\subsection{Sensitivity}\label{sensitivity}

Most of functional traits had a significant long term influence on
ecosystem outputs (figure \ref{fig:ftVar}). Only specific maximum
diameter dmax add higher diversity for greater orders implying better
evenness in species distributions. Regarding functional composition,
traits fixed to mean did not change other functional traits density
distribution. Moreover, seedrain did not seem to affect aboveground
biomass and final ecosystem height and diameter structure. Seedrain
constant fixed to 2 or 20 seed per hectare seemed to have a similar
effect. Lower seedrain implied faster decrease of stem above 10 cm dbh
and higher number of stem above 30 cm diameter at breast height after
ecosystem resilience to disturbance (approximately 50 years). Lower
seedrain than default decreased basal area over time. In addition, lower
seedrain than default decreased equitability by increasing abundance of
abundant species and decreasing abundance of less abundant species. See
\protect\hyperlink{appendix-4-sensitivity-analysis}{Appendix 4:
Sensitivity analysis} for further details.

\begin{figure}[htbp]
\centering
\includegraphics{master-thesis_files/figure-latex/ftVar-1.pdf}
\caption{\label{fig:ftVar}Functional traits effect on simulation ecosystem
variations over time. Number of trees with dbh above 10 cm (N10) and 30
cm (N30), above ground biomass (AGB) and basal area (BA). Sensitivity of
model to functional traits was performed by fixing species trait values
to their mean. Grey area represents the interval of control replicates
whereas black line represents the mean of control replicates, thus if
ecosystems outputs are outside of grey area values the studied parameter
is considered to have a significant influence on the model.}
\end{figure}

\subsection{Disturbance}\label{disturbance-1}

\subsubsection{Ecosystem functions}\label{ecosystem-functions-1}

We transformed all ecosystem outputs from the 240 disturbance
simulations in resilience metrics normalizing the treatment values by
their corresponding control (Figure \ref{fig:datatrans} A et B). We then
gathered ecosystem outputs by main ecosystem functions (forest dynamic
and forest production) to compute ecosystem distance to equilibrium
(Figure \ref{fig:datatrans} C). Finally, we integrated distance to
equilibrium in a cummulative sum over time (Figure \ref{fig:datatrans}
D).

\begin{figure}[htbp]
\centering
\includegraphics{master-thesis_files/figure-latex/datatrans-1.pdf}
\caption{\label{fig:datatrans}Ecosystem outputs data transformation.
Ecosystem outputs (\textbf{A}) are normalized by the control value over
time to calculate resilience (\textbf{B}); resilience of different
ecosystem outputs is then used in a multidimensional space to caluclate
ecosystem distance to equilibrium (\textbf{C}); finally distance ot
equilibirum is integrated over time in a cummulative sum (\textbf{D}).}
\end{figure}

The ranking was stable over time for the 240 simulations. So we used the
cummulative integral after 600 years \(Ieq_{600}\) as a measurement of
ecosystem resilience. We compared cummulative integral after 600 years
to communities taxonomic and functional diversity for each leavel of
disturbance (see Figure \ref{fig:gIeq}). We found that increased
functional diversity \citep[FDiv,][]{villeger_new_2008} was reducing
cummulative integral from ecosystem distance to forest structure
equilibrium after 600 years (\(Ieq_{600}\)). In addition, functional
evenness was complementary reducing \(Ieq_{600}\). Finally species
richness was not directly link to \(Ieq_{600}\). Effectivelly, low
species richness could result in variant \(Ieq_{600}\), but increased
species richness resulted in increased functional diversity and
consequently lower \(Ieq_{600}\). We found similar results for all
disturbance levels and forest functionning (see
\protect\hyperlink{appendix-5-disturbance-simulations}{Appendix 5:
Disturbance simulations}).

\begin{figure}[htbp]
\centering
\includegraphics{master-thesis_files/figure-latex/gIeq-1.pdf}
\caption{\label{fig:gIeq}Ecosystem resilience after 600 years with taxonomic
and functional diversity for different levels of disturbance.
Cummulative integral from ecosystem distance to forest structure
equilibrium after 600 years was represented against functional diversity
\citep[FDiv,][]{villeger_new_2008} for different level of disturbance
(25, 50 and 75\% of total basal area); dot shapes represents the species
richness whereas dot color represents functional evenness
\citep[FEve,][]{villeger_new_2008}.}
\end{figure}

\subsubsection{Biodiversity effect}\label{biodiversity-effect-1}

We measured all ecosystem outputs biodiversity net effect in disturbance
simulations by comparing them to their species corresponding monoculture
simulations. The net effect was then partition between selection and
complementarity effect. We normalized complementarity and selection
effect of disturbed simulations by biodiversity net effect of
undisturbed control simulation (see Table \ref{tab:tNE}) to measure the
resilience to disturbance of their aboveground biomass (see Figure
\ref{fig:gBE}). We found that complementarity effect was recovering
biodiversity net effect and was stronger than selection effect in the
first decades. But after few decades the complementarity effect
diminsihed toward a low value. On the contrary selection effect was
reduced or even removed by the disturbance. It increased during the
whole simulation and was greater than complementarity effect only after
decades. The time lag for which complementarity effect was greater than
selection effect was increasing with disturbance intensity. Finally 600
years after the disturbance event biodiversity net effect was still not
recovered for a disturbance intensity greater than 25\% of basal area.
We obtained those results for aboveground biomass. We found similar
results but with an amplified signal for basal area (\(BA\)) and stem
abundance (\(N\) but with an inverted signal because of the forest self
thinning) (see
\protect\hyperlink{appendix-5-disturbance-simulations}{Appendix 5:
Disturbance simulations}). Finally, forest growth primary productivity
(\(GPP\)) recovered in few years (proportionnaly to disturbance), and
its net effect was maintained by complementarity effect (see
\protect\hyperlink{appendix-5-disturbance-simulations}{Appendix 5:
Disturbance simulations}).

\begin{longtable}[]{@{}lrrll@{}}
\caption{\label{tab:tNE}Table 1: Biodiversity net effect mean value and
standard deviation for different ecosystem variable.}\tabularnewline
\toprule
variable & mean & standard deviation & name & unit\tabularnewline
\midrule
\endfirsthead
\toprule
variable & mean & standard deviation & name & unit\tabularnewline
\midrule
\endhead
agb & 32.721 & 17.839 & aboveground biomass &
\(tonC.ha^{-1}\)\tabularnewline
ba & 1.633 & 0.743 & basal area & \(m^2.ha^{-1}\)\tabularnewline
n & 103.880 & 231.623 & number of stems & \(n.ha^{-1}\)\tabularnewline
n10 & -6.815 & 13.003 & number of stems above 10cm dbh &
\(n.ha^{-1}\)\tabularnewline
gpp & 0.147 & 0.047 & growth primary production &
\(MgC.ha^{-1}\)\tabularnewline
npp & -0.041 & 0.036 & net primary production &
\(MgC.ha^{-1}\)\tabularnewline
Rday & 0.046 & 0.018 & autotrophic respiration during day &
\(MgC.ha^{-1}\)\tabularnewline
Rnight & 0.074 & 0.030 & autotrophic respiration during night &
\(MgC.ha^{-1}\)\tabularnewline
\bottomrule
\end{longtable}

\begin{figure}[htbp]
\centering
\includegraphics{master-thesis_files/figure-latex/gBE-1.pdf}
\caption{\label{fig:gBE}Resilience of complementarity and selection effects.
Complementarity effect (CE) and selection effect (SE) where normalized
by control net effect (NEc), thus measuring their resilience over time.}
\end{figure}

\subsection{Sylviculture}\label{sylviculture-1}

We repeated design of experiments and simulations used for the
disturbance simulations with the sylviculture module. But only 37
simulations included harvestable species. Moreover most of simulations
included low harvestable volume resulting in a small disturbance from
0.5 to 2.0 \(m^2.ha^{-1}\) (see Figure \ref{fig:gIeqSylv}). Consequently
most of disturbed volume was due to primary and secondary tracks and was
not related to logged trees. Thus sylviculture module resulted in
aggregated localized gaps but not in species selective disturbance. We
found that, contrarily to previous results, functional richness was
significantly (\(p<0.01\)) decreasing forest resilience see Figure
\ref{fig:gIeqSylv}). But the range of variation from resilience was
really low (\(2 < I_{eq_{600}} < 10\)). And resilience was mainly due to
the density of \emph{Bocoa prouacensis} in the simulated mature forests.
Effectively \emph{Bocoa prouacensis} was often the main harvested
species and possessed additionnally a quick recovery inside the model
due to unrealisticaly advantageous parameters. Still we found that
functional dispersion was decreasing resilience but not significantly
(see \protect\hyperlink{appendix-6-sylviculture-simulations}{Appendix 6:
Sylviculture simulations}). Finally, biodiversity net effect is almost
undisturbed due to the low disturbance intensity from the sylviculture
module and results did not show complementarity or selection effect in
forest resilience (see
\protect\hyperlink{appendix-6-sylviculture-simulations}{Appendix 6:
Sylviculture simulations}).

\begin{figure}[htbp]
\centering
\includegraphics{master-thesis_files/figure-latex/gIeqSylv-1.pdf}
\caption{\label{fig:gIeqSylv}Ecosystem resilience after 600 years with
taxonomic and functional diversity. Cummulative integral from ecosystem
distance to forest structure equilibrium after 600 years normalized by
disturbed basal area was represented against functional richness
\citep[FRic,][]{villeger_new_2008}. Dot color represents the species
richness (nb) whereas dot size represents the disturbed basal area
(m2/ha). Grey line represents the linear regression and grey area the
confidence interval.}
\end{figure}

\section{Discussion}\label{discussion}

In the limit of the model, we were able to show that diversity improved
tropical forest resilience. More particularly, functional diversity and
evenness are key components of diversity in forest recovery after
disturbance. Moreover, we found that complementarity between species was
insuring forest recovery in forest succession start with facilitation
before more productive species dominate the forest and insure recovery.
Our results advocates for a sustainable harvesting of tropical forests
through an increased resilience due to high diversity. But this
conclusion should met a sustainable definiton of selective logging
following \citet{Zimmerman2012}. Because if the harvesting is not
sustainable negative feedbacks will slowly diminish diversity and its
benefits for forest resilience, resulting in forest degradation.

\subsection{TROLL limits}\label{troll-limits}

Belowground processes, herbaceaous plants, epyphytes and lianas are not
simulated in TROLL but they not reprensent the only limit of the model.
Other processes are simulated but simplified, and we used sensitivity
analysis to assess their relative importance. We found that few
functional traits were influencing whole forest structure and dynamic.
In addition, we found that the seed rain constant had an important
effect on species functional composition and diversity. High external
seed rain resulted in a quick recovery of the system toward an
equilibrium close to the regional species frequency levied by the seed
rain. On the contrary, low seed rain let the simulated forest works as a
closed system with more system feedback but a lower stability through
time with longer species diversity transitions. In order to study the
role of diversity in forest resilience we decided to remove the seed
rain constant to get a closed system and look at the role of diversity
when it maintains itself through feedbacks and not with immigration.

Finally, sylviculture module implemented inside TROLL showed lacks of
ability to reproduce selective logging with current design of
experiments. Results did not seem to be usable and will not be
discussed. The main issues was the ability of the model to simulate
mature forest with correct abundancy of mature trees from commonly
harvested species. Two solutions might be possible in the tunning of the
model. First, forest inventories could be used to initialize the model
instead of a bare soil resulting in realistic species abundances,
especially for harvested species. Secondly, selective logging could
focus on guilds of species meeting peculiar values of functional traits
\citep[as wood density, see][]{Huth2004, Khler2004, Ruger2008} allowing
the model to harvest more volume in order to meet reality.

\subsection{Diversity improve tropical forest
resilience}\label{diversity-improve-tropical-forest-resilience}

Our results validated the hypothesis of a significative relationship
between forest resilience and functional diversity and evenness
(\(p < 0.01\) and \(p < 0.05\) respectivelly). Thus we were able to show
that diversity improve forest resilience. More particularly, functional
diversity seems a major aspect of resilience if its strengthened by a
high functional evenness. Effectively high functional diversity needs
evenness in order to better answer disturbance, if not the diversity is
masked by ecosystem dominant species. Those results confirms the review
of \citet{Diaz2001} advocating for underevaluated importance of plant
functional diversity in ecosystem processes. Additionally, the role of
evenness confirm the review of \citet{Zhang2012} who highlighted the
role of evenness in productivity and thus resilience following our
hypothesis. Finally, species and functional richnesses are not directly
increasing resilience. But increased species and functional richnesses
will increase chance for high functional diversity through the sampling
effect \citep{Loreau1998}. An higher sampling of regional species pool
will allow a greater chance to pick more functionally diverse species.

\subsection{Complementarity and selection insure forest
resilience}\label{complementarity-and-selection-insure-forest-resilience}

We found that complementary effect was insuring forest resilience in the
beginning of forest successions. We interpreted this results as the
consequence of facilitation processes. As the only ressource simulated
by TROLL is the light, the main facilitation will be light shading of
post-pioneer species by pioneer species in disturbance gaps. Our results
confirm the study of \citet{Morin2011} who also highlighted the
importance of facilitation through light shading in forest resilience.
But the complementarity effect is reducing through time, to let the
selection effect insure forest resilience due to an inforced dominance
of more productive species. The diminishing of complementarity effect is
due to competitive selection though time in gaps succession. But the
study scale matters as shown by \citet{Chisholm2013}. Here we look at
processes to a 16 ha scale. We do not have topography reducing
micro-environment effects, but \citet{Chisholm2013} results suggest that
complementarity will be stronger at smaller scale and could explain its
low value after forest recovery. Finally, our findings confirm results
of \citet{Tobner2016} realized on experimental forests of 4 years in low
diverse forest of Canada. They also found that selection effect was
greater than complementarity in most of the cases. In the case of
selective logging the complementarity effect will thus be the major
effect between two cutting cycles due to the cycle length (several
decades in guyana shield). High forest diversity with important species
complementarity through increased functional diversity is thus an
advantage for forest recovery between two cutting cycles in order to
maintain both productive ecosystem and sustainable management.

\subsection{Conclusion}\label{conclusion}

We used closed forest system simulated with TROLL forest model to
evaluate the role and mechanisms of biodiversity in tropical forest
resilience to disturbance. We found that diversity improved forest
resilience together with productivity \citep{Liang2016}. Additionally we
found that complementarity between species was insuring forest recovery
in forest succession start with facilitation, before more productive
species dominate the forest and insure forest recovery. Our results
advocate for sustainable selective logging against monoculture stand:
high diversity will increase forest resilience and thus improve logging
cycles. Even if monoculture stand are naturalized they will not reach
mature forest diversity, and consequently they will not reach its
natural high resilience. But on the other hand selective logging in
tropical forest needs to be sustainable, if not the diversity will
slowly decrease after each cycles degrading forest ecosystem functions,
and resilience will decrease due to negative feedbacks.
\citet{Zimmerman2012} criticized the state of sylviculture in tropics,
suggesting that ``we have not been able to reconcile these opposing
biological and economic forces''. Still, they advocates for a
possibility of sustainable tropical logging, already existing in
small-scale, that will need proper funding from the internationnal
communtiy. This view correspond to the high resilience of tropical
forest hyperdiverse systems, we were able to show, and let hope for a
future sustainable selective logging in tropical forests.

\appendix


\hypertarget{appendix-1-troll-model}{\section{Appendix 1: TROLL
model}\label{appendix-1-troll-model}}

In this Appendix we further detail modules of TROLL model.

\subsection{Abiotic environment}\label{abiotic-environment}

A voxel space, with a resolution of 1 \(m^3\), is used to explicilty
model the abiotic environment. For each tree crown, leaf area density is
calculated on tree geometry assuming a uniform distriution across voxels
occupied by the crown. Leaf area density is computed within each voxel
summing all tree crowns inside the voxel \(v\), and is denoted
\(LAD(v)\) (leaf area per voxel in \(m².m^{-3}\)). The vertical sum of
\(LAD\) from voxel \(v\) to the ground level defines \(LAI(v)\) (leaf
area per fround area in \(m^2.m^{-2}\) commonly called leaf area index):

\begin{equation}
  LAI(v) = \sum _{v'=v} ^\infty LAD(v') 
  \label{eq:LAI}
\end{equation}

Daily variations in light intensity (photosynthetic photon flux density
PPFD in \(\mu mol_{photons}.m^{-2}.s^{-1}\)), temperature (T in degrees
Celsius), and vapor pressure deficit (VPD in \(kPA\)) are computed to
assess carbon assimilation within each voxel of the canopy and for a
representative day per month (see Appendix 1 from \citet{Li} for further
details). Variation of PPFD Within the canopy is calculated as a loacal
Beer-Lambert extinction law:

\begin{equation}
  PPFD_{max,month}(v) = PPFD_{top,max,month}*e^{-k*LAI(v)}
  \label{eq:PPFD}
\end{equation}

The daily maximum incident PPFD at the top of canopy
\(PPFD_{top,max,month}\) is given as input. The extinction rate \(k\) is
assumed as constant, besides is variation with zenith angle and species
leaf inclination angle \citep{Meir2000}. Moreover only vertical light
diffusion is considered ignoring lateral light diffusion, which can have
an important role especially in logging gaps. Finally, intra-day
variation at half hour time steps \(t\) for a representative day every
month are used to compute \(PPFD_{month}(v,t)\), \(T_{month}(v,t)\) and
\(VPD_{month}(v,t)\). Water and nutrient process both in soil and inside
trees are not simulated.

\subsection{Photosynthesis}\label{photosynthesis}

\subsubsection{Theory}\label{theory}

Troll simulates the carbon uptake of each individual with the Farquhar,
von Caemmerer and Berry model of C3 photosynthesis \citep{Farquhar1980}.
Gross carbon assimilation rate (\(A\) in
\(\mu mol~CO_2. m^{-2}.s^{-1}\)) will be the minimum of eiter Rubisco
activity (\(A_v\)) or RuBP generation (\(A_j\)):

\begin{equation}
  A=min(A_v, A_j)~|~A_v=V_{cmax}*\frac{c_i-\Gamma^*}{c_i+K_m}~;~A_j=\frac{J}{4}*\frac{c_i-\Gamma^*}{c_i+2*\Gamma^*}
  \label{eq:A}
\end{equation}

\(V_{cmax}\) is the maximum rate of carboxylation
(\(\mu mol~CO_2.m^{-2}.s^{-1}\)). \(c_i\) is the \(CO_2\) partial
pressure at carboxylation sites. \(\Gamma^*\) is the \(CO_2\)
compensation point in absence of dark respiration. \(K_m\) is the
apparent knietic constant of the Rubisco. And \(J\) is the electron
transport rate (\(\mu mol e^-.m^{-2}.s^{-1}\)). \(J\) depends on the
light intensity with \(PPFD\):

\begin{equation}
  J = \frac{1}{2*\theta}*[\alpha*PPFD+J_{max}-\sqrt{(\alpha*PPFD+J_{max})^2}-4*\theta*\alpha*PPFD*J_{max}]
  \label{eq:J}
\end{equation}

\(J_{max}\) is the maximal electron transport capacity
(\(\mu mol e^-.m^{-2}.s^{-1}\)). \(\theta\) is the curvature factor. And
\(\alpha\) is the apparent quantum yield to electron transport
(\(mole^-.mol~photons^{-1}\)).

Carbon assimilation by photosynthesis will then be limited by the
\(CO_2\) partial pressure at carboxylation sites. Stomata controls the
gas concentration at carboxylation sites throught stomatal transport:

\begin{equation}
  A = g_s*(c_a-c_i)
  \label{eq:Ag}
\end{equation}

\(g_s\) is the stomatal conductance to \(CO_2\)
(\(molCO_2.m^{-2}.s^{-1}\)). TROLL simulates stomatal conductance
\(g_s\) with the model from \citep{Medlyn2011}:

\begin{equation}
  g_s = g_0 + (1 + \frac{g_1}{\sqrt{VPD}})*\frac{A}{c_a}
  \label{eq:gs}
\end{equation}

\(g_0\) and \(g_1\) are parameters from the model. TROLL model assume
\(g_0 \approx 0\) (empirically tested and considered as reasonable).

\subsubsection{Parametrization}\label{parametrization}

Leaf traits can be used as proxy of photosynthesis, especially leaf
nutrient content which directly play a role in it
\citep{wright_worldwide_2004}. \citet{Domingues2010} suggested that
\(V_{cmac}\) and \(J_{max}\) were both limited by the leaf concentration
of nitrogen \(N\) and phosphorus \(P\) (\(mg.g^{-1}\)):

\begin{equation}
  log_{10} V_{cmax-M} = min( 
  \begin{array}{c} 
    -1.56+0.43*log_{10} N-0.37*log_{10} LMA \\
    -0.80+0.45*log_{10} P-0.25*log_{10} LMA 
  \end{array} 
  )
  \label{eq:VcmaxM}
\end{equation}

\begin{equation}
  log_{10} J_{max-M} = min(
  \begin{array}{c} 
    -1.50+0.41*log_{10} N-0.45*log_{10} LMA \\
    -0.74+0.44*log_{10} P-0.32*log_{10} LMA 
  \end{array}
  )
  \label{eq:JmaxM}
\end{equation}

\(V_{cmax-M}\) and \(J_{max-M}\) are the photosynthetic capacities at
\(25^\circ C\) of mature leaves per leaf dry mass (resp.
\(\mu mol CO_2.g^-1.s^{-1}\) and \(\mu mol e^-.g^{-1}.s^{-1}\)). \(LMA\)
is the leaf mass per are (\(g.cm^{-2}\)). \(V_{cmax}\) and \(J_{max}\)
are calculated by multiplying \(V_{cmax-M}\) and \(J_{max-M}\) by
\(LMA\). \(V_{cmax}\) and \(J_{max}\) variation with temperature are
caluclated with \citet{Bernacchi2003} (see Appendix 2 from \citet{Li}
for further details).

TROLL computes leaf carbon assimilation \(A_l\) combining equations from
\eqref{eq:A} to \eqref{eq:JmaxM} for each tree crown voxel within in each
crown layer \(l\):

\begin{equation}
  A_l = \frac{1}{n_v*t_M} * \sum_v  \sum^{t_M}_{t=1} A(PPFD_{month}(v,t),VPD_{month}(v,t),T_{month}(v,t))
  \label{eq:Al}
\end{equation}

\(PPFD_{month}(v,t)\), \(VPD_{month}(v,t)\) , and \(T_{month}(v,t)\) are
derived from microclimatic data. \(n_v\) is the number of voxels within
crown layer \(l\). And the sum is calculated over the \(t_M\)
half-hourly intervals \(t\) of a tipical day.

\subsection{Autotrophic respiration}\label{autotrophic-respiration}

A large fraction of plants carbon uptake is actually used for plant
maintenance and growth respiration. The autotrophic respiration can
represents up to 65\% of the gross primary productivity but varies
strongly among species, sites, and environnements.

TROLL uses \citet{Atkin2015} database of mature leaf dark respiration
and associated leaf traits to compute leaf maintenance respiration:

\begin{equation}
  R_{leaf-M} = 8.5431-0.1306*N-0.5670*P-0.0137*LMA+11.1*V_{cmax-M}+0.1876*N*P
  \label{eq:Rl}
\end{equation}

\(R_{leaf-M}\) si the dark respiration rate per leaf dry mass at a
temperaure of \(25^\circ C\) (\(nmolCO_2.g^{-1}.s^{-1}\)). The other
terms are in equations \eqref{eq:VcmaxM} and \eqref{eq:JmaxM}. TROLL assume
leaf respiration during day light to be 40\% of leaf dark respiration,
and computes total leaf respiration by accounting for the legnth of
daylight.

TROLL model stem respiration (\(R_{stem}\) in \(\mu molC.s^{-1}\)) with
a constant respiration rate per volume of sapwood:

\begin{equation}
  R_{stem} = 39.6*\pi*ST*(dbh-ST)*(h-CD)
  \label{eq:Rs}
\end{equation}

dbh, h, CD and ST are tree diameter at breast height, height, corwn
depth and sapwoond thickness, respectively (\(m\)). TROLL assumes
\(ST=0.04~m\) when \(dbh>30~cm\) and an increasing \(ST\) for lower
\(dbh\).

Finally, TROLL computes both fine root maintenance respiration, as half
of the leaf maintenance respiration. Whereas coarse root and branch
maintenance respiration is computed as half of the stem respiration. And
growth respiration (\(R_{growth}\)) is assumed to account for 25\% of
the gross primary productivity minus the sum of maintenance
respirations.

\subsection{Net carbon uptake}\label{net-carbon-uptake}

Net primary production of carbon for one individual \(NPP_{ind}\)
(\(gC\)) is computed by the balance between gross primary production
\(GPP_{ind}\) and respirations \(R\):

\begin{equation}
  NPP_{ind} = GPP_{ind} - R_{maintenance} - R_{growth}
  \label{eq:NPP}
\end{equation}

TROLL partitions individuals total leaf area \(LA\) into three pools for
different leaf age classes corresponding to different photosynthesis
efficiency (young, mature and old leaves with \(LA_{young}\),
\(LA_{mature}\), and \(LA_{old}\) respectivelly). Consequently we can
compute growth primary production for one individual as:

\begin{equation}
  GPP_{ind} = 189.3 * \Delta t * \sum _{l= \lfloor h-CD \rfloor +1} ^{\lfloor h \rfloor} [A_l] * (\frac{LA_{young}}{2} + LA_{mature} + \frac{LA_{old}}{2})
  \label{eq:GPP}
\end{equation}

h and CD are tree height and crown depth, repectivelly (\(m\)).
\(\lfloor x \rfloor\) is the rounding function. \(\Delta t\) is the
duration of a timestep (\(year\)).

Thus, TROLL can compute carbon allocation to wood into an increment of
stem volume \(\Delta V\) (\(m^3\)):

\begin{equation}
  \Delta V = 10^{-6} * \frac{f_{wood}*NPP_{ind}}{0.5*wsg}*Senesc(dbh)
  \label{eq:DeltaV}
\end{equation}

\(f_{wood}\) is the fixed fraction of NPP allocated to stem and
branches. \(wsg\) is the wood specific gravity (\(g.cm^{-3}\), see
\ref{tab:traits}). TROLL assume large trees less efficient to convert
NPP as growth by using a size-related growth decline with function
\(Senesc\) after a specific diameter at brest height threshold
\(dbh_{thresh}\):

\begin{equation}
  Senesc(dbh) = max(0;3-2*\frac{dbh}{dbh_{thresh}})
  \label{eq:Senesc}
\end{equation}

Finally, TROLL can compute carbon allocation to canopy with canopy NPP
fraction denoted \(f_{canopy}\) and decomposed into leaf, twig and fruit
production. Carbon allocation to leaf results in a new young leaf pool,
whereas other leaf pools are updated as follow:

\begin{equation}
  \begin{array}{c} \\
   \Delta LA_{young} = \frac{2*f_{leaves}*NPP_{ind}}{LMA}-\frac{LA_{young}}{\tau_{young}} \\
  \Delta LA_{mature} = \frac{LA_{young}}{\tau_{young}} - \frac{LA_{mature}}{\tau_{mature}}\\
  \Delta LA_{old} = \frac{LA_{mature}}{\tau_{mature}} - \frac{LA_{old}}{\tau_{old}}
  \end{array}
  \label{eq:DeltaLA}
\end{equation}

\(\tau_{young}\), \(\tau_{mature}\), and \(\tau_{old}\) are species
residence times in each leaf pools (\(years\)). The sum of residency
time thus defined the leaf lifespan
\(LL = \tau_{young} + \tau_{mature} + \tau_{old}\) (\(years\)).
\(\tau_{young}\) is set to one month and \(\tau_{mature}\) is set to a
third of leaf lifespan \(LL\). Belowground carbon allocation is not
simulated inside TROLL.

\subsection{Tree growth}\label{tree-growth}

Once the increment of stem volume \(\Delta V\) calculated with equation
\eqref{eq:DeltaV}, TROLL convert it into an increment of tree diameter at
breast height denoted \(\Delta dbh\). TROLL infer tree height from
\(dbh\) using a Michaelis-Menten equation:

\begin{equation}
  h = h_{lim}*\frac{dbh}{dbh + a_h}
  \label{eq:h}
\end{equation}

On the other hand, we have the trunk volume
\(V = C * \pi * (\frac{dbh}{2})^2*h\), thus:

\begin{equation}
  \begin{array}{c} \\
    \Delta V = C*\frac{1}{2}*\pi*h*dbh*\Delta dbh + C * \pi * (\frac{dbh}{2})^2*h \\
    \Delta V = V*\frac{\Delta dbh}{dbh}*(3-\frac{dbh}{dbh + ah})
  \end{array}
  \label{eq:Deltadbh}
\end{equation}

Next, TROLL used the new trunk dimension (\(dbh\) and \(h\)) to update
tree crown geometry using allometric equations \citep{Chave2005}:

\begin{equation}
  \begin{array}{c} \\
    CR = 0.80 + 10.47*dbh - 3.33*dbh^2\\
    CD = -0.48 + 0.26*h~;~CD = 0.13 + 0.17*h~(h<5~m)
  \end{array}
  \label{eq:C}
\end{equation}

Finally, TROLL computes the mean leaf density within the crown (\(LD\)
in \(m^2.m^{-3}\)) assuming a uniform distribution:

\begin{equation}
  LD = \frac{LA_{young}+LA_{mature}+LA_{old}}{\pi*CR^2*CD}
  \label{eq:LD}
\end{equation}

\subsection{Mortality}\label{mortality}

Mortality is partitioned in three factors inside TROLL: background death
\(d_b\), treefall death \(d_t\) and negative density dependent death
\(d_{NDD}\). Because density dependent death \(d_{NDD}\) is still in
development inside TROLL we did not used it, so we will not detail is
computation.

\citet{chave_towards_2009} advocated for a wood economics spectrum
opposing fast growing light wood species species with high risk of
mortality to slow growing dense wood species with reduced risk of
mortality. Hence, background mortality is derived from wood specific
gravity \(wsg\) inside TROLL:

\begin{equation}
  d_b = m*(1-\frac{wsg}{wsg_{lim}})+d_n
  \label{eq:db}
\end{equation}

\(m\) (\(events.year^{-1}\)) is the reference background death rate for
lighter wood species (pioneers). \(d_n\) represents death by
carbohydrates shortage. If the number of consecutive day with
\(NPP_{ind} < 0\) \eqref{eq:NPP} is superior to tree leaf lifespan \(d_n\)
is set to 1 and remains null in other cases.

Mortality by treefall inside TROLL depends on a specifric stochastic
threshold \(\theta\):

\begin{equation}
  \theta = h_{max}*(1-v_T*|\zeta|)
  \label{eq:theta}
\end{equation}

\(h_{max}\) is the maximal tree height. \(v_T\) is the variance term set
to 0.3. \(|\zeta|\) is the absolute value of a random centered and
scaled Gaussian. If the tree hieght \(h\) is superior to \(\theta\) then
the tree may fall with a probability \(1-\theta/h\) \citep{Chave1999}.
The treefall direction is random (drawn from a uniform law
(\(\mathcal{U}[0,2\pi]\)). All tree in the trajectory of the falling
tree will be hurted through a variable denoted \(hurt\), incremented by
fallen tree height \(h\). If a tree height is inferior than its \(hurt\)
values then it may die with a probability
\(1-\frac{1}{2}\frac{h}{hurt}\). \(hurt\) variable is reset to null at
each timestep (\(month\)).

\subsection{Recruitment}\label{recruitment}

Once the tree became fertile they will start to disperse seeds. TROLL
consider tree as fertile after a specific height threshold
\(h_{mature}\) \citep{Wright2005}:

\begin{equation}
  h_{mature} = -11.47+0.90*h_{max}
  \label{eq:hmature}
\end{equation}

But TROLL is not considering seed directly through a seedbank, instead
seed might be interpreted as a seedling recruitment opportunity. The
number of reproduction opportunities per mature tree is denoted \(n_s\)
and set to 10 for all species. This assumption originates from a
trade-off between seed number and seed size resulting in equivalent
survival and recruitment probability. All \(n_s\) events are dispersed
with a distance randomly drawn from a Gaussian distribution.
Additionally, TROLL model consider external seedrain through \(n_{ext}\)
events of seed immigration:

\begin{equation}
  n_{ext} = N_{tot}*f_{reg}*n_{ha}
  \label{eq:next}
\end{equation}

\(N_{tot}\) is the external seedrain per hectare (number of reproduction
opportunities). \(f_{reg}\) is the species regional frequency.
\(n_{ha}\) is the simulated plot size in \(ha\).

Finally, a bank of seedlings to be recruited is defined for each pixel.
If the ground-level light reaches a species light compensation point
\(LCP\) the species will be recruited:

\begin{equation}
  LCP = \frac{R_{leaf}}{\phi}
  \label{eq:LCP}
\end{equation}

\(R_{leaf}\) is the leaf respiration for maintenance (see \eqref{eq:Rl}).
\(\phi\) is the quantum yield (\(\mu mol C.\mu mol~photon\)) set to
0.06. If several species reach their \(LCP\), one is picked at random.
Seedlings are recruited with following intial geometry:

\begin{equation}
  \begin{array}{c} \\
    dbh = \frac{a_h}{h_{max} - 1}\\
    h = 1~m\\
    CR = 0.5~m\\
    CD = 0.3~m\\
    LD = 0.8~m^2.^{-3}
  \end{array}
  \label{eq:C}
\end{equation}

\hypertarget{appendix-2-leaf-lifespan-model}{\section{Appendix 2: Leaf
lifespan model}\label{appendix-2-leaf-lifespan-model}}

TROLL model previous implementation encompass Reich's 1991 and 1997 and
Wright's 2004 allometries to estimate leaf lifespan with
\citep{Reich1991a, Reich1997, wright_worldwide_2004}. But we have shown
that Reich's allometries are underestimating leaf lifespan for low LMA
species. Moreover simulations estimated unrealistically low aboveground
biomass for low LMA species. We assumed Reich's allometries
underestimation of leaf lifespan for low LMA species being the source of
unrealistically low aboveground biomass inside TROLL simulations. We
decided to find a better allometry with \citet{wright_worldwide_2004}
GLOPNET dataset.

\subsection{Material and methods}\label{material-and-methods-1}

We compiled functional traits from GLOPNET
\citep{wright_worldwide_2004}, TRY \citep{Kattge2011}, and DRYAD
\citep{chave_towards_2009} databases (see \ref{tab:A2traits}). We kept
dataset given by GLOPNET as origin dataset for observations. Dataset
defined as origin corresponded to leaf lifespan (\(LL\)) and most of the
time to leaf mass per area (\(LMA\)) and leaf nitrogen content per leaf
dry mass (\(Nmass\)). We measured variable importance in functionnal
traits to explain leaf lifespan with an out-of the bag method applied on
a random forest. Then, we used a bayesian apporach to test different
models with growing level of complexity. We retained the model with the
best tradeoff between model complexity (number of parameters \(K\)),
convergence, likelihood, and prediction quality (root mean sqaure error
of prediciton \(RMSEP\)). We finally tested the new allometry obtained
with the selected model with TROLL simulations.

\begin{longtable}[]{@{}lllr@{}}
\caption{\label{tab:A2traits}Functional traits gathered with
TRY.}\tabularnewline
\toprule
Name & Trait & Unit & TRYcode\tabularnewline
\midrule
\endfirsthead
\toprule
Name & Trait & Unit & TRYcode\tabularnewline
\midrule
\endhead
LL & Leaf lifespan (longevity) & month & 12\tabularnewline
SLA & Leaf area per leaf dry mass (specific leaf area, SLA) &
\(m^2.kg^{-1}\) & 11\tabularnewline
N & Leaf nitrogen (N) content per leaf dry mass & \(mg.g^{-1}\) &
15\tabularnewline
P & Leaf phosphorus (P) content per leaf dry mass & \(mg.g^{-1}\) &
14\tabularnewline
wsg & Stem dry mass per stem fresh volume (stem specific density) &
\(mg.mm^{-3}\) & 4\tabularnewline
\bottomrule
\end{longtable}

\subsection{Results}\label{results-1}

Out of the bag method applied on a random forest highlighted the
importance of leaf nitrogen content per leaf dry mass (\(N_m\)) to model
leaf lifespan (see \ref{tab:A2OOB}). \(N_m\) importance was higher than
leaf mass per area (158 against 96 percent of mean square error
increase) which was used as a proxy for leaf lifespan in previous
models. Finally, wood specific gravity (\(wsg\)) add also a significant
importance in leaf lifespan estimation.

\begin{longtable}[]{@{}lrr@{}}
\caption{\label{tab:A2OOB}Variable importance calculated with out-of the bag
method applied on a random forest. First column represents the mean
decrease in mean square error (\%IncMSE) whereas second column
represents the total decrease in node impurities, measured by the Gini
Index (IncNodePurety). Leaf lifespan (LL) is taken in GLOPNET database
from \citet{wright_worldwide_2004}. Leaf mass per area (LMA), and leaf
nitrogen content (Nmass) are taken both in TRY
(\url{https://www.try-db.org}) and GLOPNET databases. Wood specific
gravity (wsg) is taken both in TRY and DRYAD databases.}\tabularnewline
\toprule
& \%IncMSE & IncNodePurity\tabularnewline
\midrule
\endfirsthead
\toprule
& \%IncMSE & IncNodePurity\tabularnewline
\midrule
\endhead
LMA & 99.69390 & 2028.079\tabularnewline
Nm & 159.03360 & 2666.670\tabularnewline
wsg & 50.97284 & 1475.023\tabularnewline
\bottomrule
\end{longtable}

The selected model had a maximum likelihood of 13.6 and a RMSEP of 12
months:

\begin{equation}
  LL_{d} \sim log\mathcal{N}({\beta_1}_d*LMA - {\beta_2}_d*N*\beta_3*wsg, \sigma)
  \label{eq:LL}
\end{equation}

Leaf lifespan \(LL\) follows a lognormal law with location infered from
leaf lifespan \(LMA\), nitrogen content \(N\) and wood specific gravity
\(wsg\) and a scale \(\sigma\). Each \({\beta_i}_d\) is following a
normal law located on \(\beta_i\) with a scale of \(\sigma_i\). All
\(\beta_i\), \(\sigma_i\), and \(\sigma\) are assumed without
presemption following a gamma law. \(d\) represents the dataset random
effects and encompass environmental and protocol variations.

\begin{figure}[htbp]
\centering
\includegraphics{master-thesis_files/figure-latex/A2LLpred-1.pdf}
\caption{\label{fig:A2LLpred}Leaflifespan predictions for the selected model
with leaf mass per area (LMA), leaf nitrogen content (Nmass), wood
specific gravity (wsg) and predicted versus observed values. Leaf
lifespan (LL) is predicted with model M10 fit. Leaf mass per area (LMA)
and leaf nitrogen content (Nmass), and wood specific gravity (wsg) are
taken in a composite dataset of GLOPNET, TRY and DRYAD datasets.Warning
LMA (resp. Nmass and wsg) is not constant and depend on the closest
point value for right (resp. center and left) graph.}
\end{figure}

Simulations are validating that this new allometry resolve the issue of
unrealistically low aboveground biomass for low LMA species due to an
early death of individuals inside simulations. For instance with this
allometry Symphonia sp 1 (a low LMA species) is now reaching a realistic
aboveground biomass above 400 \(tonC.ha^{-1}\) and realistic diameter
and age distribution inside the final population.

\hypertarget{appendix-3-rotten-tree-model}{\section{Appendix 3: Rotten
tree model}\label{appendix-3-rotten-tree-model}}

In order to simulate sylviculture with TROLL we needed to implement a
new sylviculture module inside TROLL model code. A first litterature
review was completed by an interview with Laurent Descroix of the Office
Nationale des Forêts. We discovered that rotten trees were not random
and seemed to depend both on tree species and diameter. This document
presents modelling of relation between rotten trees and their species
and diameter.

In fact we have two different objectives:

\begin{itemize}
\tightlist
\item
  Predict if a tree will be probed as rotten (models \textbf{M})
\item
  Predict how much of tree volume is rotten (models \textbf{N})
\end{itemize}

First all \textbf{M} model can be written as follow:
\[Rotten_n \sim \mathcal{B}(\theta_n), ~~n \in [1,N_{=3816}] ~~ p \in [1, P_{=8}], ~~ s \in [1, S_{=43}]\]
Secondly, all \textbf{N} models depend on a latent variable being the
percentage of rotten wood \(Pt_r\). We can assume that all trees are
growing depending on species \(s\) and plot \(p\) fertility and are
supposed to have a full healthy volume \(V_h\) for a given diameter
\(dbh\). We obtain following model:

\[V_f \sim log\mathcal{N}(V_h*Pt_r, \sigma), ~~n \in [1,N_{=3268}] ~~ p \in [1, P_{=8}], ~~ s \in [1, S_{=43}]\]

We retained following models :

\begin{longtable}[]{@{}ll@{}}
\caption{\label{tab:A3models}Models summary.}\tabularnewline
\toprule
M & Model\tabularnewline
\midrule
\endfirsthead
\toprule
M & Model\tabularnewline
\midrule
\endhead
\(M_{s,p}\) &
\({P_{rotten}}_n \sim \mathcal{B}(inv_{logit}(\beta_0 + \beta_1*dbh_n + {\beta_2}_p + {\beta_3}_s))\)\tabularnewline
\(N_{s,p} + L_{s,p}\) &
\(Volume_{of~wood} \sim log\mathcal{N} (log[((\beta + \beta_p + \beta_s)*dbh^2)*(1 - Pr*((\theta + \theta_p + \theta_s) *dbh^2))], \sigma)\)\tabularnewline
\bottomrule
\end{longtable}

\subsection{Probed rotten (M)}\label{probed-rotten-m}

Based on complexity (number of parameters), convergence and likelihood
we selected model \(M_{p,s}\):

\(M_{s,p}\):
\({P_{rotten}}_n \sim \mathcal{B}(inv_{logit}(\beta_0 + \beta_1*dbh_n + {\beta_2}_p + {\beta_3}_s))\)

\includegraphics{master-thesis_files/figure-latex/A3Mpred-1.pdf}

\begin{longtable}[]{@{}lrrrrrrrrrrrrrrrrr@{}}
\caption{\label{tab:A3Mtab}Models prediction. Probability to be probed as
rotten (P in \%) for a given dbh (cm).}\tabularnewline
\toprule
& 45 & 50 & 55 & 60 & 65 & 70 & 75 & 80 & 85 & 90 & 95 & 100 & 105 & 110
& 115 & 120 & 125\tabularnewline
\midrule
\endfirsthead
\toprule
& 45 & 50 & 55 & 60 & 65 & 70 & 75 & 80 & 85 & 90 & 95 & 100 & 105 & 110
& 115 & 120 & 125\tabularnewline
\midrule
\endhead
P & 4 & 4 & 5 & 7 & 8 & 10 & 12 & 14 & 17 & 20 & 24 & 28 & 32 & 37 & 42
& 47 & 52\tabularnewline
\bottomrule
\end{longtable}

\subsection{Rotten volume (N)}\label{rotten-volume-n}

Based on complexity (number of parameters), convergence and likelihood
we selected model \(N_{p,s}\) associated to hyperparameter \(\rho\) with
model \(L_{p,s}\):

\(N_{s,p} + L_{s,p}\):
\(Volume_{of~wood} \sim log\mathcal{N} (log[((\beta + \beta_p + \beta_s)*dbh^2)*(1 - Pr*((\theta + \theta_p + \theta_s) *dbh^2))], \sigma)\)

\includegraphics{master-thesis_files/figure-latex/A3Npred-1.pdf}

\begin{longtable}[]{@{}lrrrrrrrrrrrrrrrrr@{}}
\caption{\label{tab:A3Ntab}Models prediction. Final volume of wood (\(V_f\)
in \(m^3\)) and percent of rotten wood (\(V_p\) in \%) for a given dbh
(cm) if the tree was probed rotten.}\tabularnewline
\toprule
& 45 & 50 & 55 & 60 & 65 & 70 & 75 & 80 & 85 & 90 & 95 & 100 & 105 & 110
& 115 & 120 & 125\tabularnewline
\midrule
\endfirsthead
\toprule
& 45 & 50 & 55 & 60 & 65 & 70 & 75 & 80 & 85 & 90 & 95 & 100 & 105 & 110
& 115 & 120 & 125\tabularnewline
\midrule
\endhead
Vf & 1.66 & 2.02 & 2.39 & 2.78 & 3.18 & 3.58 & 3.98 & 4.37 & 4.74 & 5.09
& 5.41 & 5.68 & 5.9 & 6.06 & 6.15 & 6.16 & 6.08\tabularnewline
Vp & 7.00 & 9.00 & 11.00 & 13.00 & 15.00 & 18.00 & 20.00 & 23.00 & 26.00
& 29.00 & 32.00 & 36.00 & 40.0 & 43.00 & 47.00 & 52.00 &
56.00\tabularnewline
\bottomrule
\end{longtable}

\hypertarget{appendix-4-sensitivity-analysis}{\section{Appendix 4:
Sensitivity analysis}\label{appendix-4-sensitivity-analysis}}

To study resistance and resilience of ecosystem face to disturbance,
highlighting the role of biodiversity, we decided to use TROLL model
simulations \citep{Chave1999}. In order to get a finer study of
simulations response, we needed to assess sensitivity of the TROLL model
to different parameters. More particularly we needed to assess the
importance of functional traits to further better control functional
diversities in simulations. We also needed to assess sensitivity of the
model to see rain constant because we assumed it was one of the main
factor of tree recruitments after disturbance in the model.

\subsection{Material and methods}\label{material-and-methods-2}

To assess the sensitivity of TROLL model to species functionnal traits,
we performed a sensitivity analysis by fixing species trait values to
their mean. Each trait was tested independently. We reduced to a common
mean traits with a correlation \(r \geq 0.8\) (see figure
\ref{fig:A4corr}).

\begin{figure}[htbp]
\centering
\includegraphics{master-thesis_files/figure-latex/A4corr-1.pdf}
\caption{\label{fig:A4corr}Correlation of functional traits within TROLL
model species Blue represents negative coorelations whereas red
represents positive correlations. Values and colour intensity represents
correlation values.}
\end{figure}

To assess the sensitivity of TROLL model to seed rain, we performed a
sensitivity analysis by fixing simulations seed rain constant to 2, 20,
200 and 2000 seeds per hectare.

Simulations were conducted on Intel Xeon(R) with 32 CPUs of 2.00GHz and
188.9 GB of memory. We assumed maturity of the forest after 500 years of
regeneration \citep{Li} and computed simulation 100 years after a
disturbance event of 40\% intensity. Due to computer limitations we did
not run replicates (besides it should be necessary to reduce simulation
stochasticity). To assess ecosystem outputs sensitivity to studied
parameters, we compared it to 100 replicates of control simulations with
all parameters set to default values. Ecosystem outputs outside of the
range of the control replicates values are significantly influenced by
the studied parameter.

\subsection{Results}\label{results-2}

\subsubsection{Control}\label{control}

Both disturbed ecosystem structure and functional composition
corresponded to ecosystem structure and functional composition before
disturbance (figure \ref{fig:A4controlStr}). Consequently, we assumed
that disturbance did not affect much ecosystem structure and function.
Secondly range of values inside control replicates is low (figure
\ref{fig:A4controlVar}).

\begin{figure}[htbp]
\centering
\includegraphics{master-thesis_files/figure-latex/A4controlStr-1.pdf}
\caption{\label{fig:A4controlStr}Ecosystem structure before disturbance and
disturbed. Ecosystem structure before disturbance (left) and disturbed
(right) with diameter structure (A, B), diversity at different orders
(C, D) and rank-abundance diagrams (E, F).}
\end{figure}

\begin{figure}[htbp]
\centering
\includegraphics{master-thesis_files/figure-latex/A4controlVar-1.pdf}
\caption{\label{fig:A4controlVar}Control replicates variation. Maximum, mean
and minimum number of trees with dbh above 10 cm (N10) and 30 cm (N30),
above ground biomass (AGB) and basal area (BA) over simulation time.}
\end{figure}

\subsubsection{Functional traits}\label{functional-traits}

Most of functional traits had a significant long term influence on
ecosystem outputs (figure \ref{fig:A4ftVar}). Only \textbf{seed volume}
was always in the range of variation of control replicates. On the other
hand, few functional traits influenced final ecosystem structure (figure
\ref{fig:A4ftStr}). Only specific maximum diameter \textbf{dmax} add
higher diversity for greater orders implying better evenness in species
distributions. Regarding functional composition, traits fixed to mean
did not change other functional traits density distribution.

\textbf{ah-hmax} traits fixed to mean increased number of stems above 10
and above 30 cm dbh and basal area after disturbance (but not in long
term) and did not affect aboveground biomass. Similarly, wood specific
gravity \textbf{wsg} trait fixed to mean had exactly the same effect on
number of stems above 10 and above 30 cm dbh and basal area after
disturbance than \textbf{ah-hmax} but with a time lag ; and \textbf{wsg}
also increased aboveground biomass. \textbf{dmax} trait fixed to mean
slightly decreased number of stems above 10 and above 30 cm dbh over
time while it increased basal area, aboveground biomass, and species
evenness. Finally, leaf mass per area \textbf{LMA} trait fixed to mean
only decreased number of stem above 10 cm dbh after ecosystem resilience
to disturbance (approximately 50 years) but did not affect other
ecosystem outputs.

\begin{figure}[htbp]
\centering
\includegraphics{master-thesis_files/figure-latex/A4ftVar-1.pdf}
\caption{\label{fig:A4ftVar}Functional traits effect on simulation ecosystem
variations over time. Number of trees with dbh above 10 cm (N10) and 30
cm (N30), above ground biomass (AGB) and basal area (BA). Grey area
represents the interval of control replicates whereas black line
represents the mean of control replicates.}
\end{figure}

\begin{figure}[htbp]
\centering
\includegraphics{master-thesis_files/figure-latex/A4ftStr-1.pdf}
\caption{\label{fig:A4ftStr}Functional traits effect on simulation ecosystem
final structure. Tree final height histogram for traits (A) and control
(B), tree final diameter histogram for traits (C) and control (D),
ecosystem final diversity plot at different orders (E), and ecosystem
final rank-abundance diagram (F).}
\end{figure}

\subsubsection{Seed rain}\label{seed-rain}

Seedrain constant affected ecosystem outputs only when set lower than
default value (figure \ref{fig:A4srVar} \& \ref{fig:A4srStr}). Moreover,
seedrain did not seem to affect aboveground biomass (figure
\ref{fig:A4srVar}) and final ecosystem height and diameter structure
(figure \ref{fig:A4srStr}). Seedrain constant fixed to 2 or 20 seed per
hectare seemed to have a similar effect. Lower seedrain implied faster
decrease of stem above 10 cm dbh and higher number of stem above 30 cm
dbh after ecosystem resilience to disturbance (approximately 50 years).
Lower seedrain than default decreased basal area over time. In addition,
lower seedrain than default decreased equitability by increasing
abundance of abundant species and decreasing abundance of less abundant
species. Seedrain constant even decreased the total number of species
when fixed to 2 seed per hectare. Finally, seedrain constant slightly
affected functional composition with higher pike on ecosystem most
representatives functional trait values. In a nutshell, the lower is the
seedrain constant the most the functional density distribution is
aggregated around few functional trait values.

\begin{figure}[htbp]
\centering
\includegraphics{master-thesis_files/figure-latex/A4srVar-1.pdf}
\caption{\label{fig:A4srVar}Seed rain effect on simulation ecosystem
variations over time. Number of trees with dbh above 10 cm (N10) and 30
cm (N30), above ground biomass (AGB) and basal area (BA). Grey area
represents the interval of control replicates whereas black line
represents the mean of control replicates.}
\end{figure}

\begin{figure}[htbp]
\centering
\includegraphics{master-thesis_files/figure-latex/A4srStr-1.pdf}
\caption{\label{fig:A4srStr}Seed rain effect on simulation ecosystem final
structure. Tree final height histogram for traits (A) and control (B),
tree final diameter histogram for traits (C) and control (D).}
\end{figure}

\subsection{Discussion}\label{discussion-1}

\subsubsection{Disturbance simulation}\label{disturbance-simulation}

Ecosystem structure, species organisation and functional composition
stayed the same before and after disturbance. We can thus validate
disturbance module in its actual state. Morevoer we can now consider
that the composition we will initialise at the beginning of simulations
will stay the same after disturbance. Finally control replicates has
shown few stochasticity, it advocates for few or no replicates in
further analysis.

\subsubsection{Functional traits
selection}\label{functional-traits-selection}

\textbf{ah-hmax} fixed to mean implied no high or low trees. Less high
trees left space for more trees increasing number of stems in the
ecosystem thus increasing basal area. Wood specific gravity \textbf{wsg}
fixed to mean mainly increased wood density of light wood species.
Globally higher wood density increased lifespan of individuals
responsible for the time lag and the higher number of stems increasing
basal area. \textbf{wsg} fiwed to mean also increased carbon capture by
individuals, thus increasing aboveground biomass. Specific maximum
diameter \textbf{dmax} fixed to mean decreased death rates. Decreased
death rate diminished number of stems, expecially big ones thus
increasing global basal area and abiveground biomass. \textbf{dmax}
fixed to mean by keeping more small diameter stems also increased random
selection of species increasing evenness in species distribution.
Considering the high correlation between \textbf{ah} and \textbf{hmax}
we could also keep only \textbf{hmax} (because of its more
straightforward ecological meaning).

\subsubsection{Seed rain constant
influence}\label{seed-rain-constant-influence}

Seedrain constant did not directly affect ecosystem global outputs over
simulation post-disturbance time but have a major effect on species and
functional composition and diversity. Reducing seedrain constant
resulted in an ecosystem selecting few species increasing their
abundance and functional dominance of their traits. Thus reduced
seedrain constant greatly diminsihed evenness untill a decrease of total
number of species for its lowest value.

\hypertarget{appendix-5-disturbance-simulations}{\section{Appendix 5:
Disturbance simulations}\label{appendix-5-disturbance-simulations}}

\subsection{Ecosystem functions}\label{ecosystem-functions-2}

This appendix presents ecosystem resilience after 600 years with
taxonomic and functional diversity for different levels of disturbance.
It encompass all functionnal diversity components \citep[FRIC, FEve,
FDiv, and FDis,][]{villeger_new_2008}. And it presents results for both
forest structure (Figure \ref{fig:A5IeqStructureAll}) and forest
functionning (Figure \ref{fig:A5IeqProductionAll}).

\begin{figure}[htbp]
\centering
\includegraphics{master-thesis_files/figure-latex/A5IeqStructureAll-1.pdf}
\caption{\label{fig:A5IeqStructureAll}Ecosystem resilience after 600 years
with taxonomic and functional diversity for different levels of
disturbance. Cummulative integral from ecosystem distance to forest
structure equilibrium after 600 years was represented against functional
diversity \citep[FRIC, FEve, FDiv, and FDis,][]{villeger_new_2008} for
different level of disturbance (25, 50 and 75\% of total basal area);
dot shapes represents the species richness.}
\end{figure}

\begin{figure}[htbp]
\centering
\includegraphics{master-thesis_files/figure-latex/A5IeqProductionAll-1.pdf}
\caption{\label{fig:A5IeqProductionAll}Ecosystem resilience after 600 years
with taxonomic and functional diversity for different levels of
disturbance. Cummulative integral from ecosystem distance to forest
functionning equilibrium after 600 years was represented against
functional diversity \citep[FRIC, FEve, FDiv, and
FDis,][]{villeger_new_2008} for different level of disturbance (25, 50
and 75\% of total basal area); dot shapes represents the species
richness.}
\end{figure}

\subsection{Biodiversity effect}\label{biodiversity-effect-2}

Figure \ref{fig:A5allBE} presents the resilience of complementarity and
selection effects for different ecosystem metrics (AGB, BA, N, GPP and
NPP).

\begin{figure}[htbp]
\centering
\includegraphics{master-thesis_files/figure-latex/A5allBE-1.pdf}
\caption{\label{fig:A5allBE}Resilience of complementarity and selection
effects. Complementarity effect (CE) and selection effect (SE) where
normalized by control net effect (NEc), thus measuring their resilience
over time for different ecosystem variables (AGB, BA, N, GPP).}
\end{figure}

\hypertarget{appendix-6-sylviculture-simulations}{\section{Appendix 6:
Sylviculture simulations}\label{appendix-6-sylviculture-simulations}}

\subsection{Ecosystem functions}\label{ecosystem-functions-3}

This appendix presents ecosystem resilience after 600 years with
taxonomic and functional diversity after selective logging. It encompass
all functionnal diversity components \citep[FRIC, FEve, FDiv, and
FDis,][]{villeger_new_2008}. And it presents results for both forest
structure (Figure \ref{fig:A6IeqStructureAll}) and forest functionning
(Figure \ref{fig:A6IeqProductionAll}).

\begin{figure}[htbp]
\centering
\includegraphics{master-thesis_files/figure-latex/A6IeqStructureAll-1.pdf}
\caption{\label{fig:A6IeqStructureAll}Ecosystem resilience after 600 years
with taxonomic and functional diversity. Cummulative integral from
ecosystem distance to forest structure equilibrium after 600 years
normalized by disturbed basal area was represented against functional
functional diversity \citep[FRIC, FEve, FDiv, and
FDis,][]{villeger_new_2008}. Dot color represents the species richness
(nb) whereas dot size represents the disturbed basal area (m2/ha). Grey
line represents the linear regression and grey area the confidence
interval.}
\end{figure}

\begin{figure}[htbp]
\centering
\includegraphics{master-thesis_files/figure-latex/A6IeqProductionAll-1.pdf}
\caption{\label{fig:A6IeqProductionAll}Ecosystem resilience after 600 years
with taxonomic and functional diversity. Cummulative integral from
ecosystem distance to forest functionning equilibrium after 600 years
normalized by disturbed basal area was represented against functional
functional diversity \citep[FRIC, FEve, FDiv, and
FDis,][]{villeger_new_2008}. Dot color represents the species richness
(nb) whereas dot size represents the disturbed basal area (m2/ha). Grey
line represents the linear regression and grey area the confidence
interval.}
\end{figure}

\subsection{Biodiversity effect}\label{biodiversity-effect-3}

Figure \ref{fig:A6allBE} presents the resilience of complementarity and
selection effects for different ecosystem metrics (AGB, BA, N, GPP and
NPP).

\begin{figure}[htbp]
\centering
\includegraphics{master-thesis_files/figure-latex/A6allBE-1.pdf}
\caption{\label{fig:A6allBE}Resilience of complementarity and selection
effects. Complementarity effect (CE) and selection effect (SE) where
normalized by control net effect (NEc), thus measuring their resilience
over time for different ecosystem variables (AGB, BA, N, GPP).}
\end{figure}

\addcontentsline{toc}{section}{References}
\bibliography{/home/sylvain/Documents/Bibliography/library.bib}
\listoftables
\listoffigures

% Last pages
  %Last page
  \newpage
  \scriptsize{
  \paragraph{Résumé :}
  Les forêts tropicales font face à de nombreuses perturbations qui représentent la troisième source mondiale d’émission de gaz à effet de serre. La déforestation et la dégradation des forêts tropicales sont responsables de l’émission de 8.26 milliards de tonnes de dioxyde de carbone par an (Pearson et al. 2017). La déforestation a retenu l’attention mondiale, mais la dégradation des forêts représente 20\% des émissions de l’Amazonie brésilienne (Asner et al. 2005). La gestion durable des forêts a été proposée comme réponse à la déforestation et la dégradation, malgré la remise en question de la durabilité de l’exploitation forestière (Zimmerman \& Kormos 2012). D’autre part, les forêts tropicales abritent plus de la moitié de la biodiversité terrestre mondiale (Scheffers et al. 2012). Par conséquent, nous avons décidé d’étudier le rôle de la biodiversité dans la réponse des écosystèmes forestiers aux perturbations, en reliant diversité et fonctionnement de l’écosystème (Loreau 2010). Nous avons utilisé l’hypothèse que lors d’une perturbation, grâce à une productivité plus forte, une forêt plus diverse aura une meilleure résilience, en se basant sur la relation positive entre biodiversité et productivité. Nous avons relié cette hypothèse aux effets de complémentarité et de sélection (Loreau \& Hector 2001b). La complémentarité est la combinaison de la partition des ressources et de la facilitation, alors que l’effet de sélection est le résultat de la sélection compétitive. Nous avons ainsi centré l’étude sur les mécanismes impliqués dans la relation entre biodiversité et résilience des écosystème forestiers par une approche par simulation afin d’appréhender les processus à long terme. Nous avons utilisé le modèle TROLL (Maréchaux \& Chave) pour simuler 60 forêts matures aux diversités taxonomiques et fonctionnelles croissantes. Nous avons perturbé toutes les forêts et mesuré la résilience de leurs fonctions écosystémiques. En outre, nous avons mesuré la résilience de l’effet net de la biodiversité que l’on a décomposé en en effets de complémentarité et de sélection. Nous avons trouvé que la diversité améliore la résilience des forêts tropicales, particulièrement au travers de la diversité et l’équitabilité fonctionnelle. De plus, nous avons montré que la complémentarité entre les espèces assurait la résilience de la forêt en début de succession avant de laisser place à l’effet de sélection. Nos résultats suggèrent la possibilité d’une gestion durable des forêts tropicales grâce à une meilleure résilience avec une plus haute diversité. Mais cette conclusion n’a de sens que si l’exploitation sélective est durable (Zimmerman \& Kormos 2012). Au contraire, une gestion non durable des forêts tropicales entraînera des rétroactions négatives diminuant lentement la diversité et donc la résilience des forêts, aboutissant ultimement à la dégradation des forêts.
  \paragraph{Mots clés :} Résilience, Biodiveristé, Exploitation sélective, Fonctionnement de l'écosystème
  \paragraph{Abstract:}
  Forest disturbances are the third worldwide source of greenhous gas. Tropical deforestation and degradation emit 8.26 billion of tons of carbon dioxyde per year (Pearson et al. 2017). Deforestation has retained much attention, but degradation from forest represents 20\% of emissions in brazilian Amazon (Asner et al. 2005). Sustainable forest management has been promoted as an answer to deforestation and degradation, besides logging sustainability has been questionned (Zimmerman \& Kormos 2012). On the other hand, tropical forest host over half of the Earth’s biodiversity (Scheffers et al. 2012). Consequently, we decided to study the role of biodiversity in forest ecosystem answer to disturbance, linking diversity to ecosystem functionning (Loreau 2010). We used the hypothesis that when a disturbance event happen, due to a higher productivity, a more diverse forest will be more resilient, based on the positive relationship between biodiversity and productivity. We linked that hypothesis to the complementarity and selection effects (Loreau \& Hector 2001b). Complementarity is the addition of ressource partitionning and facilitation, whereas selection effect is the result of competitive selection. We thus focused on mechanisms involved in the relationship between biodiversity and forest ecosystem resilience with a simulation approach to assess long term processes. We used TROLL model (Maréchaux \& Chave) to simulate 60 matures forests with growing taxonomic and functional diversities. We disturbed all forests and measured the resilience of their ecosystem functions. Additionnally, we measured biodiversity net effect resilience partitioned into complementarity and selection effects. We found that diversity improved tropical forest resilience, particularly through functional diversity and eveness. Moreover, we showed that complementarity between species insured forest recovery in the beginning of the succession before being replaced by selection effect. Our results suggest the possibility for a sustainable management of tropical forest due to an increased resilience with an higher diversity. But this conclusion has meaning only if selective logging meet sustainability (Zimmerman \& Kormos 2012). On the contrary, unsustainable tropical forest management will lead to negative feedbacks slowly diminishing diversity and thus forest resilience, resulting ultimately in forest degradation.
  \paragraph{Keywords:}Resilience, Biodiversity, Selective logging, Ecosystem functionning
  }
  
  \vspace*{\fill}
  \includegraphics{images/logo}

\end{document}
